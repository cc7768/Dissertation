%!TEX root = ../../dissertation.tex

\section{A recursive two-country economy}
\label{sec:model}

We study an exchange version of the \citet{Backus1994-bi}
two-country business cycle model.
Like their model, ours has
two agents (one for each country),
two intermediate goods (``apples'' and ``bananas''),
and two final goods (one for each country).
Unlike theirs, ours has
(i)~exogenous output of intermediate goods,
(ii)~a unit root in productivity,
(iii)~recursive preferences,
and (iv)~stochastic volatility in productivity growth in one country.
The first is for convenience.
The second allows us to produce realistic asset prices.
We focus on the last two, specifically their role in the fluctuations in
consumption and exchange rates.


{\textit Preferences.\/} We use the recursive preferences developed by
\citet{Epstein1989-mt,Kreps1978-gf}, and \citet{Weil1989-vy}.
Utility from date $t$ on in country $j$ is denoted $U_{jt}$.
We define utility recursively with the time aggregator $V$:
\begin{eqnarray}
    U_{jt} &=& V [c_{jt}, \mu_{t} (U_{jt+1})]
            \;\;=\;\; [(1-\beta) c_{jt}^{\rho} + \beta \mu_{t} (U_{jt+1})^{\rho}]^{1/\rho} ,
            \label{eq:time-agg}
\end{eqnarray}
where $c_{jt}$ is consumption in country $j$ and $\mu_t$ is a certainty equivalent function.
The parameters are $ 0 < \beta < 1$ and $\rho \leq 1$.
We use the power utility certainty equivalent function,
\begin{eqnarray}
    \mu_{t} (U_{jt+1} ) &=&  \big[ E_t  (U_{jt+1}^{\alpha} ) \big]^{1/\alpha} ,
%            \;\;=\;\; \Big[ \sum_{s_{t+1}} \pi(s_{t+1} | s^t) U_{j} (s^{t+1})^{\alpha} ) \Big]^{1/\alpha}
            \label{eq:cert-equiv}
\end{eqnarray}
where $E_t$ is the expectation conditional on the state at date $t$
and $\alpha \leq 1$.
Preferences reduce to the traditional additive case when $\alpha = \rho$.


Both $V$ and $\mu$ are homogeneous of degree one (hd1).
The two functions together have the property that if consumption is constant at $c$ from date $t$ on,
then $U_{jt} = c$.

In standard terminology, $ 1/(1-\rho)$ is the intertemporal elasticity of substitution (IES)
(between current consumption and the certainty equivalent of future utility)
and $1-\alpha$ is risk aversion (RA) (over future utility).
The terminology is somewhat misleading, because changes in $\rho$ affect future utility,
the thing over which we are risk averse.
As in other multi-good settings, there's no clean separation between risk aversion
and substitutability.

{\textit Technology.\/}
Each country specializes in the production of its own intermediate good,
``apples'' in country 1 and ``bananas'' in country 2.
In the exchange case we study, production in country $j$ equals its exogenous productivity:
\begin{eqnarray}
    y_{jt} &=&  z_{jt} .
            \label{eq:production}
\end{eqnarray}
Intermediate goods can be used in either country.
The  resource constraints are
\begin{eqnarray}
    a_{1t} + a_{2t} &=& y_{1t}
            \label{eq:resource-a} \\
    b_{1t} + b_{2t} &=& y_{2t}  ,
            \label{eq:resource-b}
\end{eqnarray}
where $b_{1t}$ is the quantity of country 2's good imported by country 1
and $a_{2t}$ is the quantity of country 1's good imported by country 2.

Agents consume final goods, composites of the intermediate goods defined by the
Armington aggregator $h$:
\begin{eqnarray}
    c_{1t} &=& h (a_{1t}, b_{1t} )
            \;\;=\;\; \big[(1-\omega) a_{1t}^{\sigma} + \omega b_{1t}^{\sigma}\big]^{1/\sigma}
            \label{eq:armington-1} \\
    c_{2t} &=& h (b_{2t} , a_{2t})
            \label{eq:armington-2}
\end{eqnarray}
with $ 0 \leq \omega \leq 1$ and $\sigma \leq 1$.
The elasticity of substitution between the two intermediate goods is $1/(1-\sigma)$.
The function $h$ is also hd1.

We will typically use $\omega < 1/2 $, which puts more weight on the home good
in the production of final goods.
This ``home bias'' in final goods production is essential.
If $\omega = 1/2$, the two final goods are the same.
In this case, we have identical hd1 utility functions,
and any optimal allocation
involves a constant Pareto weight and proportional consumption paths.

{\textit Shocks.\/}
Fluctuations in this economy reflect variation in
the productivities cum endowments $z_{jt}$ and the conditional variance of the first one.
Logs of productivities have unit roots and are cointegrated:
\begin{eqnarray}
    \left[
    \begin{array}{c}
    \log z_{1t+1} \\ \log z_{2t+1}
    \end{array}
    \right]
    &=&
    \left[
    \begin{array}{c}
    \log g \\ \log g
    \end{array}
    \right] +
    \left[
    \begin{array}{cc}
    1-\gamma & \gamma \\ \gamma & 1-\gamma
    \end{array}
    \right]
    \left[
    \begin{array}{c}
    \log z_{1t} \\ \log z_{2t}
    \end{array}
    \right] +
    \left[
    \begin{array}{c}
    v_t^{1/2} w_{1t+1} \\ v^{1/2} w_{2t+1}
    \end{array}
    \right] ,
    \label{eq:lom-z}
\end{eqnarray}
with $0 < \gamma < 1/2$.
The only asymmetry in the model is the conditional variance
of productivity.
The conditional variance $v$ of $\log z_{2t+1}$ is constant.
The conditional variance $v_t$ of $\log z_{1t+1}$ is AR(1):
\begin{eqnarray}
        v_{t+1} &=& (1-\varphi_v) v + \varphi_v v_t + \tau w_{3t+1} .
        \label{eq:lom-v}
\end{eqnarray}
The innovations $\{ w_{1t}, w_{2t}, w_{3t} \} $ are standard normals
and are independent of each other and over time.
