%!TEX root = ../../dissertation.tex

\section{Open questions}


We have documented the behavior of the Pareto weight in
a relatively simple environment.
We showed, as others have, that recursive preferences
can change the quantitative properties of the model in useful ways.
The behavior of exchange rates, in particular, is much
different from the additive case.

Beyond this, we are left with a number of open questions:
\begin{itemize}
\item What parameters govern the persistence of the Pareto weight?
Can we be more precise about the impact
of risk aversion and intertemporal substitution on the dynamics of the Pareto weight?
Of the substitutability of foreign and domestic goods
in the Armington aggregator?
\item How would this change in a production economy?
Production offers opportunities to respond to changes in exogenous variables,
particularly changes in risk.
If production in one country becomes more risky, do we shift production
to the other country?
Does capital flow to the less risky country?
Are the magnitudes plausible?
\item How do changes in the Pareto weight generated by frictions and recursive preferences compare?
Are they similar or different?
Are the two mechanisms complements or substitutes?
\end{itemize}
