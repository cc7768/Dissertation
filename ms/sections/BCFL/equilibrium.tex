%!TEX root = ../../dissertation.tex

\section{Properties of the exchange economy}
\label{sec:properties}

We compute an accurate global solution to the scaled Pareto problem
by methods described in Appendix \ref{app:computation_BCFL}.
We  describe its properties here.
We start with the Pareto weight, then go on to explore
the dynamics of the Pareto weight,
the connection between consumption and the real exchange rate,
and the responses of consumption and other variables to changes in various state variables.


{\textit The Pareto frontier.\/}
One of the outputs of the numerical solution is the value function $J$, a function of
promised utility $U$ and the exogenous state variables.
Given values for the exogenous state variables, this gives us the Pareto frontier:
the highest utility of agent 1 ($J$) consistent with a given level of utility
for agent 2 ($U$) and the productive capacity of the economy.

We describe the Pareto frontier in Figure \ref{fig:pareto-frontier}
with state variables $z_{1t} = z_{2t} = \wh{z}_t = 1$ and $v_t = v$.
The outer curve in the figure is the consumption frontier,
which echoes Figure \ref{fig:consumption-frontier}.
The inner curve is the Pareto frontier.
We see that it has much the same shape.
It's inside the consumption frontier largely because of risk:
utility is below consumption because risk reduces utility.
If we increase risk aversion to 50 ($\alpha = -49$, not shown),
it shifts in further.

Changes in the state variables change both frontiers.
Movements in productivity and output shift the frontiers --- both of them ---
in and out ($\bar{z}_t$, which affects the two intermediate goods proportionately)
or twist them ($\wh{z}_t$, which affects the two goods differently).
Changes in risk twist the Pareto frontier, since risk affects the two goods differently,
but not the consumption frontier, since it has no effect on quantities of intermediate goods.


{\textit Dynamics of the Pareto weight.\/}
We see the impact of recursive preferences in Figure \ref{fig:exchange-pareto-weight-two},
where we graph $\log \lambda^*_t $ against time for a (very long) simulation of the model.
The flat horizontal line refers to the additive case ($\alpha = \rho = -1$).
As we know, the Pareto weight doesn't change in this case.
The other line refers to the recursive case, and we see clear variation in the Pareto weight.
We also see that the variation is both large and very persistent.
%Persistence is important, as we'll see, in producing persistent movements in
%the real exchange rate.

This touches on a question that's been discussed extensively:
Is the Pareto weight stable, or does one agent eventually consume everything?
\citet{Anderson2005-of} and \citet{Borovicka2016-rr} document some of the difficulties
of establishing stability in similar one-good settings.
\citet{Colacito2011-zp} prove stability in a two-good
world with elasticities of substitution between goods and over time equal to one.
\citet{Colacito2013-yq}, \citet{Kollmann2015-sy}, and \citet{Tretvoll2011-lo,Tretvoll2015-lo,Tretvoll2018-dj}
solve similar models numerically and report that the solutions are stable.
We also find that they're stable, but extremely persistent.

We get a sense of how stability works in Figure \ref{fig:change-pareto-weight-ra},
where we plot the expected change in the log Pareto weight against its level.
The exogenous state variables here have been set equal to their means.
In the additive case, the expected (and actual) change is zero.
The log Pareto weight is a martingale with no variance.
With greater risk aversion, mean reversion becomes evident.
If the Pareto weight is below its steady state value of one ($\log \lambda_t^* = 0$),
it's expected to increase.
If above, it's expected to decrease.
The effect is stronger when we increase risk aversion to 50.
There is also an evident nonlinearity in the solution,
as there is in \citet[Figure 5]{Colacito2011-zp},
but most of it occurs in regions of the state space we rarely reach.


The elasticity of substitution between foreign and domestic intermediate goods
also plays a role in persistence.
See Figure \ref{fig:change-pareto-weight-arm}.
With smaller values, mean reversion is slower.
And with larger values, it's faster.
As the elasticity increases, the line flattens out and we approach
the one-good world with a constant Pareto weight.

The intertemporal elasticity of substitution also has an effect,
but with the numbers we've chosen the effect is smaller.
See Figure \ref{fig:change-pareto-weight-ies}.
Evidently smaller values of $\rho$, and larger values of the IES [$1/(1-\rho)$],
lead to flatter lines.


{\textit Consumption and exchange rate.\/}
We noted earlier that the relation between the log consumption ratio [$\log (c_{2t}/c_{1t})$]
and the log of the real exchange rate [$ \log e_t = \log (p_{2t}/p_{1t}) $]
is mediated by the log Pareto weight ($\log \lambda^*_t$).
See equation (\ref{eq:cons-rer}).
In the additive case, the Pareto weight is constant and we have a
perfect linear relationship between the two variables.
We see exactly this in the line in Figure \ref{fig:exchange-cons-rer-two}.


The scatter of points in the same figure represents the recursive case,
where the Pareto weight acts like a wedge from the perspective of the additive model.
With our numbers, the variation in the Pareto weight is enough to change
a negative correlation of minus one between the consumption ratio and exchange rate
to a slight positive correlation.
\citet{Colacito2013-yq}, \citet{Kollmann2015-sy}, and \citet{Tretvoll2011-lo} show the same.
If we increase risk aversion $1-\alpha$ to 50,
the correlation becomes strongly positive.
In the recursive model,
we can produce any correlation we like by varying the risk aversion
parameter.


Recursive preferences also have an impact on exchange rate dynamics
as the persistence in the Pareto weight is reflected in the real exchange rate.
We see in Figure \ref{fig:rer-acfs} that the additive model is much less persistent:
The half-life (where the autocorrelation function equals one-half) is about a year.
By five years, the autocorrelation is essentially zero.
Exchange rate dynamics reflect, in this case, the modest persistence of relative productivity $\wh{z}_t$.
With recursive preferences, the exchange rate is much more persistent.
In fact with these parameter values, it's virtually a martingale.
We can reproduce any level of persistence we like by varying risk aversion
between the two cases.
There's a range of opinion, summarized nicely by \citet{Crucini2008-mi},
about how much persistence we need for the model to be realistic.
The larger point is that recursive preferences are a device that can deliver
persistence in real exchange rates and macroeconomic variables in general.
It's what an older literature would call a propagation mechanism.


{\textit Responses to productivity and volatility shocks.\/}
We get another perspective on the model's dynamics from impulse responses.
Starting at the steady state, we increase one of the exogenous state variables
by one standard deviation at date one, simulate the model for several periods,
and compute the mean dynamics of all the variables in the model.
This goes somewhat beyond traditional impulse responses in linear models in
which the subsequent innovations are turned off.

In Figure \ref{fig:irf-zhat}
we describe responses to an increase in (the log of) relative productivity $\wh{z}_t$.
The effect on future values of $\log \wh{z}_t$ declines at a constant rate
as described by equation (\ref{eq:lom-zhat}).
Neither average productivity $\bar{z}_t$ nor volatility $v_t$ change,
so this implies an increase in $\log z_{1t}$ and an equal decrease in $\log z_{2t}$.
The quantity of apples goes up, and the quantity of bananas goes down.
Because of home bias, consumption goes up in country 1 (the apple eaters)
and down in country 2 (banana eaters).
The exchange rate rises as scarce bananas become more expensive.

All of this would be true in the additive case as well.
What's different is the response of the Pareto weight.
It goes up as we compensate the agent in country 2 with promises of higher future consumption.
This effect eventually wears off, but it does so very slowly.


In Figure \ref{fig:irf-v}
we describe the responses to an increase in volatility $v_t$.
Here there's no change in the quantities of intermediate goods.
In the additive case, there would be virtually no effect.
In the recursive case, utility falls, but it falls more for country 1 because
of its home bias in favor of the good whose supply has become riskier.
The social planner responds by decreasing the Pareto weight
on agent 2.
Consumption therefore rises in country 1 and falls in country 2.
The real exchange rate falls.
This is entirely a demand-side effect.
By increasing the weight on agent 1, the demand for apples goes up
and the demand for bananas goes down.
The magnitudes are small, but it's an interesting effect that we would like to explore
further in a production economy, where supply can respond to
changes in market conditions.
