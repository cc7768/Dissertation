%!TEX root = ../../dissertation.tex

\section{Introduction}

We explore the effects of recursive preferences and risk
in an otherwise standard two-country exchange economy.
We focus on the behavior of the (relative) Pareto weight,
which characterizes consumption allocations across countries.
When preferences are additive over time and across states, as they typically are,
the Pareto weight is constant in frictionless environments.
But when preferences are recursive
and agents consume different goods,
the Pareto weight can fluctuate even in frictionless environments.
This variation in the Pareto weight acts like a wedge from the perspective
of an additive model.
Among the potential byproducts are changes in the behavior
of consumption and the exchange rate.

The natural comparison is with models that use capital market frictions
to account for the anomalous features of the standard  model.
\citet{Baxter1995-nl}, \citet{Corsetti2008-lx},
\citet{Heathcote2002-aw}, \citet{Kehoe2002-lq},
and \citet{Kose2006-os}
are well-known examples.
The frictions in these papers might be viewed as devices to produce variation in the Pareto weight,
which is then reflected in prices and quantities.
In the language of \citet{Chari2007-kq},
variations in Pareto weights would appear as wedges in the frictionless model.
The question for both approaches is whether these wedges are similar to those we observe when we
confront frictionless models with evidence.
Ultimately we would like to understand how the two approaches compare,
but for now we're simply trying to understand the behavior of the Pareto weight in models
with recursive preferences.


We build in an obvious way on earlier work
with multi-good economies by
\citet{Colacito2013-yq,Colacito2011-zp,Colacito2014-ph,Kollmann2015-sy},
and \citet{Tretvoll2011-lo,Tretvoll2015-lo,Tretvoll2018-dj}.
We show how their models work and introduce some modest extensions.
We also build on the fundamental work on recursive risk-sharing by
\citet{Anderson2005-of}, \citet{Borovicka2016-rr},
and \citet{Collin-dufresne2015-gt}.
These papers study one-good worlds,
and in that respect are simpler than work with multi-good international models,
but they lay out the mathematical structure of recursive risk-sharing problems.
The last paper in the list also describes an effective computational method
that we adapt to our environment.

One byproduct is a clearer picture of what drives the dynamics
of the Pareto weights.
In many one-good economies, Pareto weights aren't stable.
Eventually one agent consumes everything.
One of the insights of \citet{Colacito2013-yq}
is that home bias and imperfect substitutability between goods
can produce stable processes for Pareto weights and consumption shares.
That's true here, as well, but we also show how changes in risk aversion
and intertemporal substitutability affect
the dynamics of the Pareto weight.
Relative to the additive case, increasing risk aversion
or decreasing intertemporal substitution
generates more persistence in the real exchange rate.
Whether this persistence is welcome depends on one's view
of the evidence.

Risk is a particularly interesting object in this context.
Random fluctuations in the relative supply of foreign and domestic goods
also affect demand --- with recursive preferences --- through their impact on future utility.
As in many dynamic models, it's not clear how (if?)
we might separate the concepts of supply and demand.
A change in the conditional variance of future endowments, however, works only though the second channel;
supplies (endowments) do not change.
Risk affects allocations of goods through its impact on the Pareto weight
without any direct impact on their supply.

All of these results are based on global numerical solutions to the Pareto problem.
These solutions take much more computer time than the perturbation methods used in most
related work,
but they come with greater assurance that the solution is accurate even
in states far from the mean of the distribution.

Notation and terminology.  We use Latin letters for variables and Greek letters for parameters.
Variables without time subscripts are means of logs.
We use the term steady state to refer to the mean of the log of a variable.
Thus steady state $x$ refers to the mean value of $\log x_t$.
