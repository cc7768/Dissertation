%!TEX root = ../../dissertation.tex

\section{Features and parameter values}


We describe some of the salient features of the model
and the benchmark parameter values we use later on.


\subsection{Features}

{\textit Consumption frontier.\/}
The resource constraints define a possibilities frontier for consumption.
We picture several examples in Figure \ref{fig:consumption-frontier}
with $z_{1t} = z_{2t} = 1$.
In each case, we compute the maximum quantity
$c_{1t}$ consistent with a given quantity $c_{2t}$,
the resource constraints (\ref{eq:resource-a},\ref{eq:resource-b}),
and the Armington aggregators (\ref{eq:armington-1},\ref{eq:armington-2}).
The shape depends on the aggregator.
When $\omega = 1/2$, the two final goods are the same and the tradeoff is linear.
When $\omega \neq 1/2$, the frontier is concave.
The degree of concavity depends on the elasticity of substitution $ 1/(1-\sigma)$
in the Armington aggregator.


In a competitive equilibrium, the slope of the consumption frontier
is (minus) the relative price of consumption in the two countries, $p_{2t}/p_{1t} = e_t$,
the real exchange rate.
From the figure, we can imagine variation in this price
produced either by moving along the frontier or by changing the frontier through
movements in the quantities of intermediate goods.

{\textit Marginal rates of substitution.\/}
Competitive equilibria and Pareto optima equate agents' marginal rates of substitution.
With recursive preferences, the intertemporal marginal rate of substitution of agent $j$ is
\begin{eqnarray}
    m_{jt+1} &=& \beta \left( \frac{c_{jt+1}}{c_{jt}}\right)^{\rho-1}
              \left( \frac{U_{jt+1}}{\mu_{t}(U_{jt+1})} \right)^{\alpha-\rho} .
              \label{eq:mrs}
\end{eqnarray}
The last term summarizes the impact of recursive preferences.
If $\alpha=\rho$, preferences are additive and the term disappears.
Otherwise anything that affects future utility can play a role in the marginal rate
of substitution, hence in allocations.
For example, a change in risk affects future utility which, in turn, alters optimal consumption allocations and market clearing prices.
Persistence is critical here, because more persistent shocks have a larger
impact on future utility.
The discount factor $\beta$ is similar:  the larger it is, the greater the weight on
future utility and the greater the impact on the marginal rate of substitution.

Note, too, the dynamics built into the recursive term.
Its log is a risk adjustment plus white noise.
The log of the numerator is
\begin{eqnarray*}
    \log U_{jt+1} &=&  E_t (\log U_{jt+1}) + \big[\log U_{jt+1} - E_t (\log U_{jt+1})\big] ,
\end{eqnarray*}
the mean plus a white noise innovation.
The log of the denominator is
\begin{eqnarray*}
    \log \mu_t (U_{jt+1})
            &=&
            \alpha^{-1} \log E_t (e^{\alpha \log U_{jt+1}})   \\
                        &=& E_t (\log U_{jt+1}) +
            \alpha^{-1} \big[ \log E_t (e^{\alpha \log U_{jt+1}}) - E_t (\alpha \log U_{jt+1}) \big] .
%            \underbrace{\mbox{risk adjustment}}_{\alpha \mbox{Var}_{t} (\log U_{t+1})/2}
\end{eqnarray*}
The term in square brackets is the entropy of $U_{jt+1}^\alpha$ and is positive.
We multiply by $\alpha$ to get what we term the risk adjustment, which is negative if $\alpha$ is.
The difference then is the innovation minus the risk adjustment.
The innovation increases the volatility of the intertemporal marginal rate of substitution,
which is the primary source of success in asset pricing
applications.

The marginal rate of substitution (\ref{eq:mrs}) is measured in units of agent $j$'s consumption good,
whose price is $p_{jt}$.
We refer to the relative price of the two consumption goods as the real exchange rate:
$ e_t = p_{2t}/p_{1t} $.
The two marginal rates of substitution are then connected by
$ m_{2t+1} = (e_{t+1}/e_t) m_{1t+1}$.
We'll derive this in the next section,
but it should be evident here that the dynamics of the exchange rate reflect the
marginal rates of substitution.
If the two marginal rates of substitution are close to white noise, as we suggested,
then the depreciation rate $e_{t+1}/e_t$ has the same property.


{\textit Productivity dynamics.\/}
One way to think about our log productivity process (\ref{eq:lom-z})
is that their average is a martingale and their difference is stable.
Denote half the sum and half the difference by
\begin{eqnarray*}
    \log \bar{z} &=& (\log z_1 + \log z_2)/2 \\
    \log \wh{z}  &=& (\log z_1 - \log z_2)/2  .
\end{eqnarray*}
Then we can express the underlying productivities
by $\log z_1 = \log \bar{z} + \log \wh{z}$ and $\log z_2 = \log \bar{z} - \log \wh{z} $.
Equation (\ref{eq:lom-z}) implies that half the sum,
\begin{eqnarray*}
    \log \bar{z}_{t+1} &=& \log g + \log \bar{z}_t + (v_t^{1/2} w_{1t+1} + v^{1/2} w_{2t+1})/2 ,
\end{eqnarray*}
is a martingale with drift.
The difference,
\begin{eqnarray}
    \log \wh{z}_{t+1}  &=& (1-2\gamma) \log \wh{z}_t + (v_t^{1/2} w_{1t+1} - v^{1/2} w_{2t+1})/2 ,
    \label{eq:lom-zhat}
\end{eqnarray}
is stable, which tells us $\log z_{1t}$ and $\log z_{2t}$ are cointegrated.
Given the linear homogeneity of the model, changes in $\bar{z}$ affect consumption
quantities proportionately, with no effect on their relative price, the exchange rate.
Changes in relative productivity $\wh{z}$, however,
affect consumption quantities differentially and therefore affect the real exchange rate
as well.

{\textit Pareto problems.\/}
We compute competitive equilibria in this environment by finding
Pareto optimal allocations and their supporting prices.
 In a two-agent Pareto problem,
we maximize one agent's utility subject to
(i)~the other agent getting at least some promised level of utility
(the promise-keeping constraint)
and (ii)~the productive capacity of the economy (the resource constraints and shocks).
The Lagrangian for this problem is
\begin{eqnarray*}
 \mathcal{L} &=& U_{1t} + \lambda_{t} (U_{2t} - \overline{U}) + \mbox{resource constraints and shocks} ,
\end{eqnarray*}
with $\lambda_t$ the multiplier on the promise-keeping constraint.
If utility functions are strictly concave,
this is equivalent to traditional Mantel-Negishi maximization of their weighted average,
\begin{eqnarray*}
    \theta_{1t} U_{1t} + \theta_{2t} U_{2t} ,
\end{eqnarray*}
with positive Pareto weights ($\theta_{1t}, \theta_{2t})$.
Evidently $\lambda_t$ in the previous problem plays the same role as $ \theta_{2t}/\theta_{1t}$.
We refer to $\lambda_t$ as the Pareto weight,
although in terms of the latter version we might call it the relative Pareto weight.


{\textit Transforming utility.\/}
We find it convenient to use an hd1 time aggregator, but with additive preferences
(the special case $\rho = \alpha$)
it's more common to transform utility to
$ {U}^{*}_{jt} = U_{jt}^\rho/\rho$.
With this transformation, equation (\ref{eq:time-agg}) becomes
\begin{eqnarray}
    {U}^{*}_{jt} &=& (1-\beta) c_{jt}^\rho/\rho
            + \beta \big[ E_t (U_{jt+1}^{* \alpha/\rho}) \big]^{\rho/\alpha} .
    \label{eq:utility-additive}
\end{eqnarray}
When $\rho=\alpha$ this takes the familiar additive form.

The transformation also changes the look of derivatives.
When we represent preferences with $U_{jt}$, marginal utility is
\begin{eqnarray*}
    \partial U_{jt} /\partial c_{jt} &=& U_{jt}^{1-\rho} (1-\beta) c_{jt}^{\rho-1} .
\end{eqnarray*}
When we use ${U}^*_{jt}$, marginal utility takes the simpler form
\begin{eqnarray*}
    \partial {U}^*_{jt} /\partial c_{jt} &=&  (1-\beta) c_{jt}^{\rho-1} .
\end{eqnarray*}
We'll use this insight later on to simplify some of the expressions we get
using the hd1 form of the time aggregator.
This includes the Pareto weight, which is defined for a specific
utility function.


\subsection{Parameter values}

We make only a modest effort to use realistic parameter values.
The goal instead is to highlight the effects of recursive preferences and
stochastic volatility with parameter values in the ballpark of those used
elsewhere in the literature.
We summarize these choices in Table \ref{tab:benchmark}.
The time interval is one quarter.

{\textit Preferences.\/}
We use $\rho=-1$ (implying an IES of one-half)
and $\alpha = -9$ (implying risk aversion of 10).
The former is a common value in business cycle modeling; \citet{Kydland1982-xy}, for example.
The latter is widely used in asset pricing; \citet{Bansal2004-mb} is the standard reference.
The key feature of this configuration is that $\alpha-\rho < 0$.
We set $\beta = 0.98$.


{\textit Technology.\/}
The Armington aggregator plays a central role here,
specifically the elasticity of substitution $1/(1-\sigma)$
between foreign and domestic intermediate goods.
A wide range of elasticities have been used in the literature.
Some earlier work used elasticities greater than one.
\citet{Colacito2011-zp,Colacito2013-yq} and \citet{Kollmann2015-sy} use an elasticity of one.
\citet[Section 3.2]{Heathcote2002-aw}, \citet[Table 1]{Tretvoll2018-dj}, and \citet[Table 3]{Tretvoll2015-lo} suggest
smaller values.
We start with an elasticity of one ($\sigma = 0$),
but consider other values, particularly when we explore the interaction
of the elasticity and the dynamics of the Pareto weight.

Given a choice of $\sigma$, we set the share parameter $\omega$ like this.
First-order conditions equate prices to marginal products:
\begin{eqnarray*}
    p_{1t} &=& c_{1t}^{1-\sigma} (1-\omega) a_{1t}^{\sigma-1} \\
    p_{2t} &=& c_{1t}^{1-\sigma} \omega b_{1t}^{\sigma-1}
\end{eqnarray*}
In a symmetric steady state with import share $s_m = b_1/(a_1 + b_1)$
and relative price $ p_{2t}/p_{1t} = 1$,
the ratio of these two equations implies
\begin{eqnarray}
    \left( \frac{1-\omega}{\omega} \right) &=& \left( \frac{1-s_m}{s_m} \right)^{1-\sigma} .
    \label{eq:share-calculation}
\end{eqnarray}
We set $s_m = 0.1$.  Given a value for $\sigma$, the import share nails down $\omega$.
One consequence of this calculation is that the parameter $\omega$ approaches one-half
as $\sigma$ approaches one (and the elasticity of substitution approaches infinity).


{\textit Shocks.}
The mean growth rate is $\log g = 0.004 $:  0.4\% per quarter.
The number comes from \citet[Table 4]{Tallarini2000-xx} and is estimated with US data.
We set the persistence parameter $\gamma$ that governs productivity dynamics equal to 0.1,
which implies an autocorrelation of $1-2\gamma = 0.8$ for $\log \wh{z}$.
\citet[Table 5]{Rabanal2011-pm} estimate $\gamma$ to be less
than 0.01, which implies significantly greater persistence.
The stochastic volatility process (\ref{eq:lom-v}) is based on \citet{Jurado2015-hy}
as described in \citet[Section 5.3]{Backus2015-yx}.
