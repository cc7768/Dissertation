%!TEX root = ../../dissertation.tex

\section{Solving the recursive Pareto problem}


We compute a competitive equilibrium indirectly as a Pareto optimum,
a standard approach in this literature.
The equilibirum prices then show up as Lagrange multipliers on the resource constraints.
Similar models have been studied by \citet{Colacito2013-yq},
\citet{Kollmann2015-sy}, and \citet{Tretvoll2011-lo}.


\subsection*{Pareto problem}

We solve the Pareto problem recursively with a slight change in notation:
We represent utility of agent 1 by $J$, the value function,
and the utility of agent 2 by $U$, without the ``2'' subscript.
The state at date $t$ is then the exogenous variables $s_t = ( z_{1t}, z_{2t}, v_t ) $
plus the utility promise $U_{t}$ made to agent 2.
The Bellman equation is
\begin{eqnarray*}
    J(U_t, s_t) &=& \max_{\{c_{1t}, U_{t+1} \}}  V \big\{ c_{1t}, \mu_t [J(U_{t+1}, s_{t+1})] \big\} \\
    \mbox{s.t.} && V \big\{ c_{2t}, \mu_t (U_{t+1}) \big\} \;\geq\; U_{t} \\
            && \mbox{plus resource constraints and shocks} .
\end{eqnarray*}
The resource constraints include (\ref{eq:resource-a},\ref{eq:resource-b}) for intermediate goods
and (\ref{eq:armington-1},\ref{eq:armington-2}) for final goods.
The shocks follow (\ref{eq:lom-z},\ref{eq:lom-v}).

We use Lagrange multipliers $\lambda_t$ on promised utility
and $(q_{1t}, q_{2t}, p_{1t}, p_{2t})$ on the resource constraints.
The first-order conditions are then
\begin{eqnarray}
    {c}_{1t}: &&  p_{1t} \;\;=\;\; {J}_t^{1-\rho} (1-\beta) {c}_{1t}^{\rho-1}
            \label{eq:foc-c1} \\
    {c}_{2t}: &&  p_{2t} \;\;=\;\; \lambda_t {U}_t^{1-\rho} (1-\beta) {c}_{2t}^{\rho-1}
            \label{eq:foc-c2} \\
    {a}_{1t}: &&  q_{1t} \;\;=\;\; p_{1t} {c}_{1t}^{1-\sigma} (1-\omega) {a}_{1t}^{\sigma-1}
            \label{eq:foc-a1} \\
    {b}_{1t}: &&  q_{2t} \;\;=\;\; p_{1t} {c}_{1t}^{1-\sigma} \omega {b}_{1t}^{\sigma-1}
            \label{eq:foc-b1} \\
    {a}_{2t}: &&  q_{1t} \;\;=\;\; p_{2t} {c}_{2t}^{1-\sigma} \omega {a}_{2t}^{\sigma-1}
            \label{eq:foc-a2} \\
    {b}_{2t}: &&  q_{2t} \;\;=\;\; p_{2t} {c}_{2t}^{1-\sigma} (1-\omega) {b}_{2t}^{\sigma-1}
            \label{eq:foc-b2} \\
    {U}_{t+1}:&&  {J}_t^{1-\rho} \beta \mu_t ({J}_{t+1})^{\rho-\alpha} {J}_{t+1}^{\alpha-1} {J}_{Ut+1}
            \;\;=\;\; - \lambda_t {U}_t^{1-\rho} \beta \mu_t ({U}_{t+1})^{\rho-\alpha} {U}_{t+1}^{\alpha-1} .
            \label{eq:foc-u}
\end{eqnarray}
Note, in particular, that equation (\ref{eq:foc-u}) applies to promises $U_{t+1}$
in every state at $t+1$.  It's many equations, not just one.

The envelope condition for ${U}_t$ is
\begin{eqnarray*}
    {J}_{Ut} &=& - \lambda_t ,
\end{eqnarray*}
which we use to replace $J_{Ut+1}$ with $\lambda_{t+1}$ in  (\ref{eq:foc-u}).

{\textit Transforming the Pareto weight.\/}
We can simplify the solution by transforming the Pareto weight $\lambda_t$.
A Pareto weight is defined for specific a utility function;
if we transform the utility function, we transform the Pareto weight along with it.
The natural benchmark for the Pareto weight is the additive case, in which it's constant.
Additive preferences are traditionally expressed using utility $U_{t}^* = U_{t}^\rho/\rho$ and $J_{t}^* = J_{t}^\rho/\rho$.
The associated Pareto weight is
\begin{eqnarray*}
    \lambda_t^* &=& \lambda_t {U}_t^{1-\rho}/ {J}_t^{1-\rho} .
\end{eqnarray*}
We refer to $\lambda^*_t$ as the additive Pareto weight ---
or simply the Pareto weight.

With this substitution, we can clearly see the impact of recursive preferences.
The first-order condition (\ref{eq:foc-u}) becomes
\begin{eqnarray}
    \beta \left( \frac{{J}_{t+1}} {\mu_t({J}_{t+1})} \right)^{\alpha-\rho}
%                  \left( \frac{e_{t+1}}{e_t} \right)
                  \lambda^*_{t+1}
            &=&  \lambda^*_t \
            \beta \left( \frac{{U}_{t+1}} {\mu_t({U}_{t+1})} \right)^{\alpha-\rho} .
            \label{eq:foc-u-additive}
\end{eqnarray}
In the additive case, $\alpha=\rho$, this reduces to
$ \lambda^*_{t+1}  = \lambda^*_{t} $ and  the Pareto weight is constant.
This is, of course, well known, but it's nice to know we're on the right track.

%*** Show one-good case reduces to constant weight. ??

With recursive preferences, the Pareto weight need not be constant, although it's an open
question how important this is quantitatively.


{\textit Consumption and the exchange rate.\/}
The same transformation of the Pareto weight changes equations (\ref{eq:foc-c1},\ref{eq:foc-c2}) to
\begin{eqnarray*}
    (1-\beta) c_{1t}^{\rho-1} / p_{1t} &=& \lambda_t^* (1-\beta) c_{2t}^{\rho-1} / p_{2t}
\end{eqnarray*}
or
\begin{eqnarray}
    p_{2t} / p_{1t} &=& \lambda_t^* (c_{2t}/c_{1t})^{\rho-1} .
    \label{eq:cons-rer}
\end{eqnarray}
If the prices of final goods are equal, as they are in a one-good world,
the first equation tells us to equate weighted marginal utilities across agents.

In the second equation, the left side is the real exchange rate $e_t$.
The right side is the product of the Pareto weight
and the consumption ratio.
In the additive case, the Pareto weight is constant and we have a linear relation between
the logs of the real exchange rate and the consumption ratio.
In the data, there is little sign of such a relation.
See, among many others, \citet{Backus1993-ul,Chari2002-ll,Corsetti2008-lx,Colacito2013-yq,Kollmann1995-fe}, and \citet{Tretvoll2011-lo}.
We might say that the data suggests a wedge between the price and consumption ratios
that is represented in the recursive model by the Pareto weight.

If we further simplify the model to have a single good,
then the real exchange rate is one in all states.
In the additive case,
$\lambda_t^*$ is constant
and equation (\ref{eq:cons-rer}) tells us that the ratio of consumptions is also constant.
Any variation in consumption by one agent is exactly mirrored by the other.
This is, of course, counterfactual, and one of the standard anomalies
of international business cycle models.


{\textit Marginal rates of substitution.\/}
The Pareto problem equates marginal rates of substitution,
but since the agents consume different goods this involves the relative price $ e_t = p_{2t} / p_{1t} $.
The ratio of equation (\ref{eq:cons-rer})
at dates $t$ and $t+1$ is
\begin{eqnarray*}
        \beta \left( \frac{c_{1t+1}}{c_{1t}}\right)^{\rho-1}
              \left( \frac{e_{t+1}}{e_t} \right)
              &=&
              \left( \frac{\lambda^*_{t+1}}{\lambda^*_t} \right)
              \beta \left( \frac{c_{2t+1}}{c_{2t}}\right)^{\rho-1} .
\end{eqnarray*}
Combining this with (\ref{eq:foc-u-additive}) gives us
\begin{eqnarray*}
        \beta \left( \frac{c_{1t+1}}{c_{1t}}\right)^{\rho-1}
              \left( \frac{J_{t+1}}{\mu_{t}(J_{t+1})} \right)^{\alpha-\rho}
              \left( \frac{e_{t+1}}{e_t} \right)
              &=&
              \beta \left( \frac{c_{2t+1}}{c_{2t}}\right)^{\rho-1}
              \left( \frac{U_{t+1}}{\mu_{t}(U_{t+1})} \right)^{\alpha-\rho}
\end{eqnarray*}
as noted earlier.
Note the role of  $\alpha - \rho$.
If the difference is zero, the recursive term drops out.
Otherwise the sign of the impact of future utility depends on the sign of $\alpha-\rho$.

%[*** Talk about relation between promises and weights?]  %??
