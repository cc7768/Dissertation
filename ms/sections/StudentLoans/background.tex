%!TEX root = ../../dissertation.tex

\section{Student Loan Examples} \label{sec:example}

In this section, we present two hypothetical examples to motivate what comes in later sections:

First, we present an example describing a single student loan borrower. We use this example to
highlight how different loan repayment plans might have an effect on the payments made by borrowers
and use our example to define the \textit{repayment burden}.

After discussing an example using a single student, we walk through an example involving an entire
cohort of individuals. We use this cohort example as a vehicle to define certain metrics which we
can use to evaluate the outcomes of loan repayment plans in our model.


\subsection{Single Individual}

  Consider a recent college graduate that accumulated \$29,000 of student loans that have a 6.8\%
  interest rate. After graduation, she begins working at a job that pays \$35,000 per year. For
  simplicity, we will assume that her income grows at 3\% annually and ignore taxes. This
  means that during her first year of employment her monthly income would be \$2,900 per month.

  Under the AMR plan, her monthly payment would be approximately \$300. The \textit{repayment
  burden} is the monthly payment divided by the monthly income. In this case, the repayment
  burden is $\frac{300}{2900} \approx 0.10$. One compelling feature of the IDR is that it places a
  strict upper bound on the repayment burden and avoids occurences of the required loan payments
  being large relative to an individual's income. In our individual's case, the payment required
  under the IDR plan would be\footnote{Disposable income is defined as 1.5 times the poverty level
  of income. For a single adult in the United States, the poverty level is around \$12,000} $0.10
  \times (2,990 - \frac{18,000}{12}) \approx 140$ which leads to a repayment burden of about
  0.05\footnote{Note that the repayment burden is not fixed at 0.10 because the IDR rate is a
  marginal cost and the repayment burden is an average cost.}.

  In Figure~\ref{fig:loan_repayment}, we see how the repayment burden evolves over time for both
  loan repayment plans. We chose our numbers so that the student in our example had the average
  amount of debt for a college graduate and the 25th percentile earnings of a recent college
  graduate. Of course, it is also important to recognize that the repayment burdens we computed
  assume that the individual maintains her employment and does not experience bouts of unemployment
  --- under the AMR as her monthly income goes to 0, the repayment burden goes to $\infty$.

  We also can see that in our example the IDR plan spreads payments over a much longer horizon (17
  years) than the AMR plan (10 years). One implication of this is that, if an individual fully
  repays her debt, then the individual will typically pay more\footnote{If their income is high
  enough that they repay their loan sooner than 10 years under IDR then they will pay less overall.}
  under IDR (\$46,500) than they would under AMR (\$37,500). Of course, if debt is forgiven at the
  end of 20 years then it is still possible that the overall payments are less.


\subsection{Cohort} \label{subsec:cohort}

  For our purposes we will define a cohort as all graduating high school students in a particular
  year. Consider a cohort that consists of 100,000 individuals. Suppose that 50,000 of these
  individuals choose to enroll in college and 25,000 successfully graduate from college. We define
  the following measures of student outcomes within a cohort:

  \begin{itemize}
    \item \textit{Enrollment rate}: The enrollment rate is defined as the number of individuals in
          a cohort who enroll in college divided by total number of individuals in cohort. In our
          example, the enrollment rate would be 0.50.
    \item \textit{Completion rate}: The completion rate is defined as the number of individuals in
          a cohort who graduate from college divided by total number of individuals in cohort. In
          the example, the completion rate would be 0.25.
    \item \textit{Graduate rate}: The graduation rate is defined as the number of individuals in a
          cohort who graduate from college divided by the number of individuals who enroll in
          college. In our example, the graduate rate would be 0.50.
  \end{itemize}

  Suppose that of the 50,000 students who enroll in school, 30,000 of the students take out some
  amount of student loans and that the total amount of loans accumulated by the cohort totals to
  \$150,000,000. We now define some measures of how much debt a particular cohort takes on while in
  college.

  \begin{itemize}
    \item \textit{Fraction in debt}: The fraction in debt is measured as the number of students who
          accumulated debt divided by the number of students who enrolled in college. In our
          example, the fraction in debt is 0.60.
    \item \textit{Average debt}: The average debt is measured as the total amount of loans
          accumulated by a cohort divided by the number of students who took out debt. In our
          example, the average debt is \$5,000.
  \end{itemize}

  Finally, suppose that over the course of the loan lifetimes, the cohort makes payments that are
  worth \$140,000,000 in present discounted value. We then compute the implied \textit{government
  subsidy}, $\Delta_{SL}$, as the difference in the amount of loans issued and the present
  discounted value of payments made by the cohort\footnote{An important note to make here is that if
  all student loans were repaid, then this subsidy would be negative. This is because the interest
  rate on student loans is typically higher than the interest rate used to compute government
  subsidies according to the Federal Credit Reform Act, so the present discounted value of the loan
  payments would exceed the cost of borrowing that money.}. In our example, the goverment subsidy is
  \$10,000,000. Furthermore, as mentioned before, the government subsidy can be composed into
  several relevant components:

  \begin{align} \label{eq:gov_subsidy}
    \Delta_{SL} &= \underbrace{\frac{\sum_i (d_i - X_i)}{\sum_i d_i}}_{\text{subsidy rate}} \times
                   \underbrace{\frac{\sum_i d_i}{N_d}}_{\text{average debt}} \times
                   \underbrace{\frac{N_d}{N_e}}_{\text{fraction in debt}} \times
                   \underbrace{\frac{N_e}{N}}_{\text{enrollment rate}} \times
                   N
  \end{align}

  where $d_i$ is the amount of student debt held be individual $i$, $X_i$ is the present discounted
  value of the payments made by individual $i$, $N_d$ is the number of students who take out loans
  while in school, $N_e$ is the number of students enrolled in college, and $N$ is the number of
  individuals in the cohort.

  The only feature of this decomposition that we have not previously seen is the \textit{subsidy
  rate}. As we see in Equation~\ref{eq:gov_subsidy}, the subsidy rate is defined as the government
  subsidy divided by the total amount of debt issued in the economy. In our example, this is
  $\frac{150,000,000 - 140,000,000}{150,000,000} \approx 0.067$.

  We will use this decomposition in Section~\ref{sec:results} to understand which channels are most
  important for understanding how the government subsidy changes when we move from AMR to IDR. This
  understanding will also help us propose alternate policies that achieve some of the risk
  mitigation benefits without incurring the extra costs implied by our model.
