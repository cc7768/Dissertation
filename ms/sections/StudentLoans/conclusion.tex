%!TEX root = ../../dissertation.tex

\section{Conclusion} \label{sec:conclusion}

As the US continues to expand their IDR programs, it is important that ED is able to provide an
accurate forecast of the required government subsidy. In their response to \cite{GAO-17-22}, they
argue that, in order to do this accurately, one must account for changes in behavior by potential
borrowers. We do this by building a structural model of pre, intra, and post college decisions and
use our model to evaluate some of the changes that might occur if the student loan program was
switched from a strictly AMR system to a strictly IDR system. We find that IDR has small effects on
student outcomes, but that it raises the cost to the government of running the student loan program.
We use our model to pinpoint what drives these increased costs and find that the increase is mostly
driven by additional debt accumulated by those who are unlikely to be able to repay their debts.
