%!TEX root = ../../dissertation.tex

\section{Repayment Plan Comparison} \label{sec:results}

Now that we have presented the model and understand which aspects of the data it successfully
replicates, we proceed to perform some counterfactuals. We will analyze how the outcomes from
Section~\ref{sec:example} differ according to the student loan repayment plan used in the economy.
In particular, we will consider three versions of our economy:

\begin{enumerate}
  \item The first, \textit{AMR}, is our baseline model and uses the AMR plan
        previously described for student loan repayment.
  \item The second, \textit{LC}, is a model which uses the income driven repayment
        plan previously described, but the individuals in the model use decision rules from the
        \textit{AMR} model.
  \item The third, \textit{IDR}, is a model which uses the income driven repayment plan for student
        loan repayment and individuals use their optimal decisions rules.
\end{enumerate}

Comparing the \textit{AMR} and \textit{IDR} versions of our economy will inform us about how the
outcomes differ across the two student student loan repayment plans. However, we think it is useful
to also include compare these outcomes with the outcomes under \textit{LC} as well because this is
how many policy analysts have previously evaluated IDR plans; they use the observed outcomes from
those who have already attended and left college and evaluate the outcomes of new repayment policies
using the decisions individuals made under previous policies\footnote{This type of policy analysis
is exactly what \cite{Lucas1976} was concerned about when he wrote, ``Given that the structure of an
econometric model consists of optimal decision rules of economic agents, and that optimal decision
rules vary systematically with changes in the structure of series relevant to the decision maker, it
follows that any change in policy will systematically alter the structure of econometric models.''}.
In contrast, comparing \textit{AMR} and \textit{IDR} accounts for the changes associated with
individuals responding to changes in their environment and, by doing this, can potentially capture
features that ED is currently missing in their forecasts.

In the subsections that follow, we will focus on comparing \textit{AMR} against \textit{IDR} and
will return to \textit{LC} in Section~\ref{subsec:idr_lc}.


\subsection{Enrollment, completion, and graduation} \label{subsec:ecg}

  We begin by thinking about the implications the different student loan repayment plans have for
  the student outcomes discussed in Section~\ref{sec:example}. Table~\ref{table:students} shows how
  these outcomes differ for the \textit{AMR} and \textit{IDR} economies. The table shows that, in
  the aggregate, the enrollment increases by 1 percentage point in \textit{IDR} relative to
  \textit{AMR} and that there are no significant changes in either the completion or graduation
  rates.

  There are two reasons that the change in enrollment rates is small

  First, there is relatively little uncertainty about an individual's probability of successfully
  graduating from college. In Figure~\ref{fig:scatter_ep_vs_p}, we confirm this by plotting the
  individual's expected probability of passing ($\int_a p(a) d F(a | j)$) against their actual
  probability of passing ($p(a)$).\footnote{The high precision of the ability signals is a result of
  \cite{HendricksLeukhina2017}}. Individuals with a low ability signal find it not to be optimal to
  attend college no matter what the repayment method is because they are relatively certain about
  their inability to graduate and, as a college dropout, they would receive the same income process
  as the high school workers but must also pay tuition and the utility cost for being a college
  dropout, $U_{\text{CD}}$.

  Secondly, enrollment rates are determined by the relative value of being a college student and
  high school worker\footnote{If the value of being a college student is given by $V^S$ and the
  value of being a high school worker is $V^{HS}$ then the discrete choice shock specification in
  our model leads the probability of enrolling to be $\frac{\exp(V^S)}{\exp(V^{HS}) + \exp(V^S)}$}.
  Since the value of being a high school worker is unaffected by the college repayment plans, the
  change in probability must come through the value of being a college student. The value of being a
  college student is affected only indirectly through the value of being a college dropout/graduate
  worker with a given repayment plan. In our model, IDR does not provide high value insurance
  because the estimated income process places low probability on individuals experiencing ``low''
  incomes. We demonstrate this in Figure~\ref{fig:vfs_enrollment} by plotting the difference in the
  value functions associated with being a college dropout in an economy under IDR and AMR across
  various income levels and we then overlay the probability distribution of these incomes for the
  individual most likely to experience low income. We continue to reiterate the caveat that there
  are aspects of labor income that do not appear in our model due to its yearly frequency. At a
  monthly frequency one could include unemployment and other features that would increase the value
  of this insurance by increasing the probability that individuals experience periods of low labor
  income, so we view these results as a lower bound on the value of this insurance.

  The fact that there is almost no change in the completion or graduation rates follows closely from
  the small change in the enrollment rates coupled with the small changes associated with the value
  of being a college dropout/graduate. Since the enrollment rate changes very little, no more than
  2\% for any type in our model, a very similar group of students choose to attend college and,
  since the value of being a college dropout/graduate has changed little, the probability that they
  dropout in any given period also changes little.


\subsection{Debt decisions} \label{subsec:dd}

  We next examine how the different repayment plans affect decisions for debt accumulation for
  college students. We see in Table~\ref{table:debt} that the fraction of college students who exit
  with debt increases from 46\% to 70\% and the average debt held by these students goes from
  \$11,750 to \$13,740. This outcome is not particularly surprising --- lowering the risk associated
  with having more debt, incentivizes students to accumulate debt while in school. However,
  understanding the composition of these changes will improve our understanding of what drives the
  change in the government subsidy of the student loan program.

  In Figure~\ref{fig:debt_by_gpa} we display the change in the fraction of students with debt and
  average debt levels by the individual's HS GPA quartile. We see that there are increases across
  the board in both variables, however, the largest effects come from a change in behavior by those
  with the lowest GPA levels. The bottom quartile sees a 50 percentage point increase in the
  fraction of students who take out loans and a \$4,000 increase in the average debt levels while
  the top quartile only sees an increase of 11 percentage points and \$1,500, respectively.

  To understand why this happens, let's now consider the marginal benefit and marginal cost of an
  additional dollar of student loans. The marginal benefit of an additional dollar of debt is some
  mix of the extra consumption and less hours worked during the period the debt is accumulated. The
  marginal cost of an additional dollar of debt is the present discounted value of the corresponding
  payments that would be made. In Figure~\ref{fig:expected_repayment}, we plot the ratio of an
  additional dollar of debt divided by the present discounted value of the payments made. Loans made
  under AMR are almost always fully recovered so the ratio is 1, but under IDR there is positive
  probability that an individual will not be required to fully repay their loan.

  Suppose for a moment that the individual was given perfect knowledge of their income realizations
  during their last year of college and that they knew that, given their current debt level,
  some amount of their loan would be forgiven. This knowledge implies
  $\frac{\partial V^{C?}_t}{\partial d_t} = 0$ which, in turn, means that the individual loses
  nothing from accumulating additional debt (but benefits from additional consumption during
  school). Obviously, this example is more dramatic than what would typically occur in our model but
  we can show that there exists some bound, $\tilde{D}$, such that as
  $\lim_{d \rightarrow \tilde{D}}$ we see
  $\frac{\partial V^{C?}_t(k_t, d_t, S_t, w_t, \varepsilon^y_t)}{\partial d_t} \rightarrow 0$. Since
  both income and GPA are positively correlated with ability level, the individuals who are least
  likely to repay their loans are those from the lowest HS GPA quartiles. This results in students
  from the lowest HS GPA quartiles having the largest increase in incentives to accumulate
  additional debt which is what we observe in our model.


\subsection{Cost of IDR}

  Finally, we turn to thinking about the relative cost of these two policies to the government. We
  write the percent change in the government subsidy associated with moving from \textit{AMR}
  to \textit{IDR} as:

  \begin{align*}
    \text{\% Change in } \Delta_{SL} &= 1 - \left( \frac{\Delta^{IDR}_{SL}}{\Delta^{AMR}_{SL}} \right) \\
  \end{align*}

  In our model, the government subsidy for the \textit{IDR} economy is 15\% larger than in the
  \textit{AMR} economy.

  We can decompose this into the four components described in Section~\ref{subsec:cohort} --- the
  enrollment rate ($\tilde{N}_e$), the fraction of students with debt ($\tilde{N}_d$), the average
  debt among students with debt ($\bar{d}$), and the subsidy rate ($\delta_{SL}$) --- to write the
  percent change as the product of the ratios of each component:

  \begin{align*}
    \text{\% Change in } \Delta_{SL} &= 1 -
      \left( \frac{\tilde{N}^{IDR}_e}{\tilde{N}^{AMR}_e} \right) \times
      \left( \frac{\tilde{N}^{IDR}_d}{\tilde{N}^{AMR}_d} \right) \times
      \left( \frac{\bar{d}^{IDR}}{\bar{d}^{AMR}} \right) \times
      \left( \frac{\delta^{IDR}_{SL}}{\delta^{AMR}_{SL}} \right)
  \end{align*}

  In Section~\ref{subsec:ecg} and Section~\ref{subsec:dd}, we've already discussed how each of these
  components changes except for the subsidy rate. In Table~\ref{table:cost}, we report the
  government subsidy rates and a per person cost\footnote{We choose to use a per person cost rather
  than report the full subsidy because the amount of subsidy is dependent on the number of
  individuals are in the simulated cohort --- If one wanted to estimate how large this subsidy would
  be for a cohort of high school graduates, this number could be multiplied by the cohort size} for
  each of the different economies.

  One unexpected feature of our model that is worth discussing is that the student loan program has
  a negative subsidy (positive returns) for the government. While there are certain components of
  the U.S. student loan portfolio that have a negative subsidy rate\footnote{For example, the
  estimated subsidy on Federal Direct Loans issued in 2017 that are being repaid using the AMR plan
  is approximately -15 billion dollars, see \cite{GAO-17-22}. If one divided this -15 billion by the
  total number of Federal Direct Loans issued that were being repaid using IDR then we would get a
  negative subsidy rate.}, this is not broadly true in the U.S. and should be seen as a shortcoming
  of our model. The negative subsidy rate is likely tied to two components of our model: First, the
  government can perfectly enforce its student loan program and there is no option for default in
  the \textit{AMR} economy. Second, everyone in our model participates in labor market for an
  extended amount of time and makes their payments. This outcome reinforces to us the fact that
  expanding the model to capture additional features of the loan repayment system or labor market
  could be useful in evaluating these proposals, but this significantly increases the computational
  difficulty of the model. Rather than get distracted with these improvements, we focus on the
  behavioral changes that occur and highlight possible channels that ED could use to improve their
  own forecasts.

  In Section~\ref{subsec:dd} we highlighted the heterogeneity in the changes of debt accumulation
  by HS GPA quartile. In Table\ref{table:subsidy_gpa}, we show how the subsidy rates change by the
  student loan repayment plan for different HS GPA quartiles. The key takeaway is that the majority
  of the increase in the required government subsidy comes from those in low HS GPA quartiles ---
  The largest of which comes from an almost 200\% increase in the required subsidy rate to students
  from the lowest GPA quartile.

  The final exercise we do is ask, ``How much could the subsidy rate decrease, holding the other
  changes that occur under \textit{IDR}, constant, without increasing the government subsidy beyond
  its \textit{AMR} value?'' In order to not increase the government subsidy, we need:

  \begin{align*}
      \left( \frac{\tilde{N}^{IDR}_e}{\tilde{N}^{AMR}_e} \right) \times
      \left( \frac{\tilde{N}^{IDR}_d}{\tilde{N}^{AMR}_d} \right) \times
      \left( \frac{\bar{d}^{IDR}}{\bar{d}^{AMR}} \right) \times
      \left( \frac{\delta^{IDR}_{SL}}{\delta^{AMR}_{SL}} \right) &= 1
  \end{align*}

  This would require a government subsidy rate of -0.07 under \textit{IDR} which is significantly
  lower than the -0.13 from \textit{AMR}. The reason that a subsidy rate of -0.07 can maintain the
  same goverment subsidy is that we have increased the volume of student loans in the economy by
  changing what fraction of students accumulate debt and how much debt they accumulate. In fact, one
  way that this subsidy rate can be achieved under the \textit{IDR} plan is by simply offering IDR
  to only the top three GPA quartiles.


\subsection{Lucas Critique} \label{subsec:idr_lc}

  We now return briefly to the \textit{LC} model. Recall that in this model we use IDR to repay the
  loans made in the economy, but the individuals in our model make decisions as if they were in the
  \textit{AMR} economy. Since the decision rules by the agents have not changed, there are no
  changes to the enrollment/dropout decisions and no changes to the amount of debt accumulated. In
  this economy, the government subsidy still increases, due to limit on the repayment burden and
  eventual forgiveness, but it only increases by 8\% rather than 15\%. We believe that this helps
  reiterate ED's point that in order to accurately evaluate the cost of the IDR plans, there must be
  an understanding of what implications they have on individual behavior.
