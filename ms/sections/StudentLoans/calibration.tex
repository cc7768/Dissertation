%!TEX root = ../../dissertation.tex

\section{Calibration} \label{sec:calibration}

The structural model laid out in Section \ref{sec:model} is dependent on many parameters and several
distributional assumptions. In an attempt to prevent ourselves from leveraging this fact too much
and to avoid overfitting the data, we use, where possible, parameter values and distributional
assumptions that have been previously used in peer-reviewed research anywhere possible. This won't
work for all of the model parameters since certain parameters are sensitive to the model
specificiation. For example, the discrete choice shocks, $\pi_e$ and $\pi_c$, need to be determined
for our particular model because the effect they have on choices is dependent on the model's Bellman
equations. For parameters that we can't determine externally, we use a simulated method of moments
approach and aim to match certain moments from the data.

We describe the data sources we use and how we computed the relevant moments in
Appendix~\ref{sec:data_appendix}.


\subsection{Distributional Assumptions}

  We discretize the types by drawing $N_j=200$ types, $j := (m^j, k_1^j, q^j, z^j)$, distributed
  according to the following distribution:

  Let $\{ \varepsilon_j^m, \varepsilon_j^k, \varepsilon_j^q, \varepsilon_j^z \}_j$ be independent
  and identically distributed standard normals\footnote{In practice, we draw the $\varepsilon$s
  using Sobol sequences rather than randomly to provide better coverage of the type space}. Then

  \begin{align*}
    k^j &= \max \{0, \mu_k + \sigma_k \frac{\varepsilon_j^k + \alpha_{km} \varepsilon_j^m}{\sqrt{1 + \alpha_{km}^2}} \} \\
    m^j &= \frac{\varepsilon_j^m + \alpha_{mq} \varepsilon_j^q + \alpha_{mz} \varepsilon_j^z}{\sqrt{1 + \alpha_{mq}^2 + \alpha_{mz}^2}} \\
    q^j &= \mu_q + \sigma_q \frac{\varepsilon_j^q + \alpha_{qz} \varepsilon_j^z}{\sqrt{1 + \alpha_{qz}^2}} \\
    z^j &= \max \{0, \mu_z + \sigma_z \varepsilon_j^z \}
  \end{align*}

  Additionally, we assume that an individual's ability, conditional on their GPA ($m$) is drawn from

  $$a \sim N(\rho m, \sqrt{1 - \rho^2})$$

  where $\rho := \frac{\alpha_{am}}{\sqrt{1 + \alpha_{am}^2}}$ is the correlation between $a$ and
  $m$. We do this by choosing $N_a=9$ discrete levels of $a$\footnote{We choose these values of $a$
  such that $a \in \{\bar{a}_1, \dots, \bar{a}_{N_a}$ with $\bar{a}_1 < \bar{a}_2 < \dots <
  \bar{a}_{N_a}$ such that each ability level is drawn with equal probability} and computing the
  implied joint probability distribution over the discrete $a, j$ pairs.

  We discretize $\zeta$ into $N_\zeta=4$ possible values. A value $\zeta_i$ determines two financial
  states of college $dq^i$ and $\Omega_v^i$. We choose the form of the per-period cost of college as
  $q(j, \zeta_i) = q_j + dq^i$. $\Omega_v$ represents the choice set of hours that an individual can
  choose to work while in school and is chosen to match various levels of part-time to
  full-time\footnote{For more details on how these values were determined, see
  \cite{HendricksLeukhina2017}}. $\zeta$ follows a Markov chain, so the cost of college and the
  hours an individual can work while in college potentially fluctuate while they are enrolled.

  We assume the age specific components of income for each education level, $\theta^s(t)$, follows a
  cubic polynomial in age as in \cite{Guvenen2009} and others

  $$\theta^s(t) = \theta^s_0 + \theta^s_1 \text{age} + \theta^s_2 \frac{\text{age}^2}{100} + \theta^s_3 \frac{\text{age}^3}{1000}$$


\subsection{Parameters}

  The model parameters are determined in three stages:


  \subsubsection{Previous Knowledge}

    The previous knowledge parameters consist of parameters that are either widely used or come from
    specific research papers. The model does not seem to be particularly sensitive to small
    deviations of these parameters.

    These parameters are listed in Table~\ref{table:pk}.


  \subsubsection{Externally Calibrated}

    Our externally calibrated parameters are parameters that have some form of a data counterpart
    that can be used to line the model up with the data. This ensures that the model will match a
    particular piece of the data, but that it cannot use these parameters to match unrelated pieces
    of data. We now briefly describe how we determined parameters that fell in this category and
    the associated values can be found in Table~\ref{table:ec}.

    We find the age specific component of income by following standard procedures for
    estimating a dynamic income process. We use PSID data on family income and restrict the sample
    to working age (24-60) males and remove those who have certain irregularities such as too low
    of yearly earnings or too many hours worked\footnote{We follow decisions made by
    \cite{Guvenen2009} and \cite{Hryshko2012} when selecting a subsample to work with}. We needed
    to compute the age specific components on our own because the income dynamics literature
    typically focuses on the residual income so these values were not reported.

    We choose to set the number of classes per year to 6 with each class worth 6 hours of
    credit\footnote{This in effect means that each course in our model corresponds to 2 courses in
    the data since most courses in college are worth 3-4 hours of credit} in the data. This means
    the maximum number of hours an individual can earn in a given year is 36 which, according to
    \cite{HendricksLeukhina2017}, is the 90th percentile of credits earned. Additionally, we set the
    maximum amount of time that students can stay in school to 6 years since less than 5\% of
    college graduates take more time than this to graduate.

    The data used in \cite{HendricksLeukhina2017} corresponds to students who were in college in the
    mid to late 1980s. We follow their lead and set the cumulative student loan limit to -\$19,750
    (in 2000 dollars) which approximately matches the 1986 limit of \$12,500.

    A list of the externally calibrated parameters can be found in Table~\ref{table:ec}


  \subsubsection{Internally Calibrated}

    The final group of parameters are parameters that do not have any data based counterpart. We
    determine values for these parameters by choosing certain features of the data that we think are
    informative about these parameters and then numerically minimize the difference between these
    features in the data and in our model.

    This includes the parameters $\delta_c$, $\delta_v$, $\pi_e$, $\pi_c$, $U_{HS}$, $U_{CD}$,
    $U_{CG}$, $\alpha_{km}$, $\mu_k$, and $\sigma_k$. We choose these jointly to match statistics on
    enrollment by HS GPA quartiles, completion by HS GPA quartiles, yearly dropout rates, yearly
    dropout rates by HS GPA quartiles, yearly fractions of college students with student loans,
    yearly average student loans, and college earnings. The parameter values for these parameters
    can be found in Table~\ref{table:ic}.


\subsection{Model Fit}

  The data and model values for the moments we described are displayed in
  Tables~\ref{table:moments}. Overall, we find that our model is able to accurately capture the
  moments of the data that we are interested in. We discuss some specific examples below:

  A mixture of the discrete choice shocks and disutility of being a college dropout/graduate helps
  us successfully match the enrollment and drop out rates quite accurately. While the overall yearly
  drop out rate matches quite well, we do slightly worse for the dropout rates by year and HS GPA
  quartile. In particular, our model predicts that those in the low GPA quartiles drop out a little
  slower than in the data, but those in the high GPA quartiles drop out a little more quickly than
  in the data. Additionally, we consider it a success that using the credit accumulation parameters
  from \cite{HendricksLeukhina2017} results in us accurately matching the educational fractions and
  completion rates.

  One feature that we struggled to match was that in our model too few people accumulated debt
  during their first year of college (11\% of students in our model had debt after the first year
  whereas the data says that 26\% of students have debt after their first year).
