%!TEX root = ../../dissertation.tex

\section{Data} \label{sec:data_appendix}

  Due to the sensitive nature of information on an individual's education data, one difficulty of
  any empirical work on higher education financing is the scarcity of publically available micro
  data. In conclusion to a discussion of how an economist might design a student loans program,
  \cite{Dynarski2014} claims, ``Designing such a program requires detailed data on individual
  earnings and borrowing, which are currently unavailable to researchers within and outside the
  government. If loan policy is to be firmly grounded in research, this gap in the data must be
  closed.'' While not as accurate as administrative data, we leverage several data sets on education
  and earnings outcomes in this paper.

  \textbf{High School and Beyond}: The High School and Beyond survey is a longitudinal survey that
  follows individuals who were either sophmores or seniors in 1980 through 1992\footnote{The
  individuals who were high school sophmores in 1980 were interviewed in 1980, 1982, 1984, 1986, and
  1992 while the individuals who were high school seniors were only interviewed in 1980, 1982, 1984,
  and 1986}. We did not have access to the restricted access micro data and associated transcript
  data, but \cite{HendricksLeukhina2017}, who did have access, published many relevant moments from
  this data set. We take advantage of this and use their empirical estimates for various parameters
  in our model.

  \textbf{Panel Study of Income Dynamics}: The Panel Study of Income Dynamics (PSID) is the longest
  running longitudinal survey in the world. As in \cite{CarrollSamwick1997}, \cite{Guvenen2009}, and
  \cite{Hryshko2012} the PSID will be used to estimate income dynamics --- We successfully replicate
  estimated income processes from these papers and will use them as inputs to our model.
