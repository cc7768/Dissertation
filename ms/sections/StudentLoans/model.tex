%!TEX root = ../../dissertation.tex

\section{Model} \label{sec:model}

Our goal is to gain insight into the benefits and costs of different student loan repayment
policies. In particular, we are interested in how the costs associated with adverse selection in the
college entrance decision, and moral hazard in the debt accumulation decision, compare with the
benefits of reduced period-by-period repayment burdens, and the improved cross-individual insurance,
that occur when using IDR.

As previously discussed, the college attendance portion of the model was originally presented in
\cite{HendricksLeukhina2017}. The novel addition is our inclusion of a risky labor market where
individuals are potentially subject to periods of low income and are forced to repay their student
loans according to the prespecified student loan plan.

We find it helpful to start with a brief outline of the model's timeline and decisions before
the formal definition of the model.


\subsection{Model Outline}

  The model is a discrete time, overlapping generations model. All individuals in the model begin as
  high school graduates. Upon graduating from high school, they observe stochastic characteristics
  that identify their idiosyncratic type and use these characteristics to decide whether to enroll
  in college or not.

  If the high school graduate chooses not to enroll in college, they immediately begin working in
  the labor market as a high school worker.

  If the high school graduate chooses to enroll in college, it follows that they become a college
  student. Each period that an individual is a college student they choose whether to continue their
  studies or to drop out. If the student drops out, they enter the labor market as a college dropout
  worker. If the student chooses to continue, they choose an amount of student loans to accumulate,
  an amount of (dis)savings, and a number of hours to work. They finance their consumption and
  education costs using their savings, student loans, and any earnings they receive from employment
  during college. They then observe their grades for the period which they use in their decision in
  the following period about whether to continue in college or dropout. If the student accumulates
  enough credits while in college, then they enter the labor market as a college graduate worker.

  Each type of worker receives risky labor income that depends on their education level and
  idiosyncratic characteristics. They all have access to non-contingent savings which can be used to
  smooth their lifetime consumption. Any individual who accumulated student loans
  is required to pay them back according to a specified student loan repayment plan.

  We now present a formal description of the model. For notational convenience, once we have
  described the idiosyncratic types that define each agent, we will describe the model in reverse
  chronological order.


\subsection{Agent Types}

  Each agent lives for $T$ periods and has preferences given by:

  \begin{align*}
    V = \sum_{t=1}^T \beta^{t-1} \left(
      \mathcal{I}_{\text{In College}} (u_c(c_t) + v_c(v_t)) +
      \mathcal{I}_{\text{Period becomes worker}} U_{s} +
      \mathcal{I}_{\text{Labor Market}} u(c_t)
    \right)
  \end{align*}

  $U_s$ is an education specific utility level with $s \in \{\text{HS}, \text{CD}, \text{CG}\}$
  where HS stands for high school worker, CD stands for college dropout, and CG stands for college
  graduate. $u_c$ and $v_c$ are the utility function for consumption and labor, respectively, during
  an individual's college life. $u$ is the worker's utility function for consumption.

  Individuals in this model graduate high school at time $t=1$. At this time, they draw three
  idiosyncratic states:

  \begin{enumerate}
    \item Ability level ($a$): An individual's ability level is distributed as $N(0, 1)$. The
    individual's ability level determines the probability with which a student passes each of their
    classes and affects their earnings once they are in the labor market. The ability level is
    unobserved until an individual enters the labor market.
    \item Type ($j$): An individual's type, $j \sim G_J$, determines a signal about their ability
    ($m$), initial wealth level ($k_1$), the base cost of attending college ($q$), and the financial
    transfers they receive from their parents for 6 periods after high school graduation ($z$).
    Additionally, we treat $m$ as unobservable to the econometrician, rather, the econometrician
    will observe a noisy signal about $m$ given by $\text{GPA} \sim N(m, \sigma^2_{\text{GPA}})$.
    \item Initial financial shock ($\zeta_1$): The financial state, $\zeta_1 \sim G_\zeta$,
    determines scholarships and work opportunities for a college student.
  \end{enumerate}


\subsection{Workers}

  Workers are subject to risky labor market income. The income process has permanent and transitory
  components and depends on an individual's ability level and the number of credits earned during
  college. At age $t$, individual $i$ with ability level $a$, earned $n$ college credits, education
  level $s$, with permanent income component $w_{i, t}$, and transitory income component
  $\varepsilon^y_{i, t}$ receives pre-tax income:

  $$y_{i, t} = exp(\theta^s(t) + \phi_s a + \mu n + w_{i, t} + \varepsilon^{s, y}_{i, t})$$

  The transitory component of income is independent and identically distributed according to
  $\varepsilon^{s, y}_{i, t} \sim N(0, \sigma^2_{s, y})$. The permanent component of income follows
  a random walk:

  \begin{align*}
    w_{i, t} &= w_{i, t-1} + \varepsilon^{s, w}_{i, t}
  \end{align*}

  where $\varepsilon^w_{i, t} \sim N(0, \sigma^2_{s, w})$. $\theta^s(t)$ is an age specific
  component of income.

  There are progressive labor income taxes in the economy, an individual who makes $y_{i, t}$
  receives $\tilde{y}_{i, t} = \lambda y_{i, t}^{1 - \tau}$.

  Workers can smooth out consumption using a risk-free asset, $k_t$, which pays the risk-free
  interest rate, $r$.


  \subsubsection{High School Workers}

    High school workers are completely described by the features above and their Bellman equation
    is given by:

    \begin{align*}
      V^{HS}_t(k_t, w_t, \varepsilon^y_t) &= \max_{k_{t+1}} u(c_t) + \beta E \left[ V^{HS}_{t+1}(k_{t+1}, w_{t+1}, \varepsilon^{HS, y}_{t+1}) \right] \\
      &\text{subject to } \\
      c_t &= (1 + r) k_t + \tilde{y}_t - k_{t+1} \\
      w_{t+1} &= w_t + \varepsilon^{HS, w}_{t+1}
    \end{align*}


  \subsubsection{College Dropout/Graduate Workers}

    College dropout and college graduate workers may have accumulated student loans during their
    time in college. These loans must be paid off during the working phase of their life according
    to the specified repayment plan. The interest rate on student loans is $r_{sd}$ and it is
    potentially different than the interest rate on the risk-free asset.

    Let $\rho(d_t, S_t, \tilde{y}_t)$ be the amount that a worker with outstanding student loans
    of $d_t$, $S_t$ payments left to make, and $\tilde{y}_t$ of after tax income is required to
    make. Let $\Gamma(d_t, S_t, \tilde{y}_t)$ be a function that determines the evolution of $S_t$.
    We allow both of these functions to depend on these three variable so that both IDR and AMR are
    special cases of the repayment function.

    We write the college dropout's/graduate's Bellman equation below:

    \begin{align*}
      V^{C?}_t(k_t, d_t, S_t, w_t, \varepsilon^y_t) &= \max_{k_{t+1}} u(c_t) + \beta E \left[ V^{s}_{t+1}(k_{t+1}, d_{t+1}, S_{t+1}, w_{t+1}, \varepsilon^{s, y}_{t+1}) \right] \\
      &\text{subject to } \\
      c_t + k_{t+1} &= (1 + r) k_t + \tilde{y}_t + \rho(d_t, S_t, \tilde{y}_t) \\
      w_{t+1} &= w_t + \varepsilon^{s, w}_{t+1} \\
      d_{t+1} &= (1 + r_{sd}) (d_t - \rho(d_t, S_t, \tilde{y}_t))
      S_{t+1} &= \Gamma_S(d_t, S_t, \tilde{y}_t)
    \end{align*}

    where $C?$ can either represent $CD$ for college dropout or $CG$ for college graduate.


\subsection{College Student}

  The life of a college student closely follows \cite{HendricksLeukhina2017}.

  Individuals who choose to enroll in college begin each period of college with $d_t$ in outstanding
  student loans, $k_t$ savings, $n_t$ college credits, $\zeta_t$ as their financial state, and are
  of type $j$. Each period an individual is enrolled in college they must first decide whether to
  dropout to become a college dropout worker or to continue as a college student. If they choose to
  continue then they choose how much to accumulate in student loans ($d_t$), their savings for the
  next period ($k_{t+1}$), and how many hours they will work ($v_t$).

  While in college, an individual's budget constraint is given by

  \begin{align*}
      c_t + CoC(q(j), \zeta_t) + k_{t+1} + d_{t+1} &\leq
      (1 + r) k_t + (1 + r_{sd}) d_t + w_{\text{coll}} v_t + z(j)
  \end{align*}

  where $c_t$ is the individual's time $t$ consumption, $CoC(q(j), \zeta_t)$ is the cost of college
  for an individual of type $j$ with financial state $\zeta_t$, and $z(j)$ are the transfers paid to
  the individual from their parents. The financial state $\zeta_t$ follows a Markov chain.

  Each period a student is enrolled, they take $n_c$ credits and pass each class with a probability
  that depends on their ability level, $p(a)$. This probability is given by

  $$p(a) = \gamma_{\text{min}} + \frac{1 - \gamma_{\text{min}}}{1 + \gamma_1 e^{-\gamma_2 a}}$$

  with $\gamma_1, \gamma_2 > 0$ which implies $\frac{\partial p(a)}{\partial a} > 0$.

  A student's college credits for the following period are given by

  $$n_{t+1} = n_t + \tilde{n}_t$$

  where $\tilde{n}_t$ is a draw from a Binomial distribution with $n_c$ draws and a probability
  $p(a)$ of success. Any student who acquires $n_{\text{grad}}$ credits graduates and becomes a
  college graduate worker.

  Recall that an individual's ability level is unknown to them. The individuals in the model are
  Bayesians and use the information about how many classes they pass each period to update their
  beliefs about their ability level. In particular, an agent of age $t$, type $j$, and who has
  accumulated $n_t$ credits so far believes that they have ability level $a_i$ with probability

  \begin{align*}
    \text{prob}(a_i | j, n_t) &= \frac{\text{prob}(j, n_t | a_i) \text{prob}(a)}{\text{prob}(j, n)} \\
    &= \frac{\text{prob}(j | a_i) \text{prob}(n_t | a_i) \text{prob}(a_i)}{\sum_{\tilde{a} \in A} \text{prob}(j, n | \tilde{a}) \text{prob}(\tilde{a})} \\
    &= \frac{\text{prob}(j | a_i) \text{prob}(n_t | a_i) \text{prob}(a_i)}{\sum_{\tilde{a} \in A} \text{prob}(j | \tilde{a}) \text{prob}(n_t | \tilde{a}) \text{prob}(\tilde{a})} \\
  \end{align*}

  We now describe the college student's problem as two Bellman equations.

  The first Bellman equation represents the college student's value of having $d_t$ in student
  loans, $k_t$ in savings, $n_t$ credits, financial state $\zeta_t$, and being type $j$ after having
  chosen to continue school in period $t$. As described earlier, they maximize with respect to their
  choice of student loans for tomorrow, savings for tomorrow, and how much to work today:

  \begin{align*}
    V^S_t(k_t, n_t, \zeta_t, j) &= \max_{d_{t+1}, k_{t+1}, v_t \in \Omega(\zeta_t)}
      u_c(c_t) + v_c(v_t) +
      \beta E \left[ V_{t+1}^{S-}(d_{t+1}, k_{t+1}, n_{t+1}, \zeta_{t+1}, j) \right] \\
    &\text{ subject to } \\
    c_t + CoC(q(j), \zeta_t) + k_{t+1} + d_{t+1} &\leq
      (1 + r) k_t + (1 + r_{sd}) d_t + w_{\text{coll}} v_t + z(j) \\
    d_{t+1} \geq \underbar{D} \\
    d_{t+1} - d_{t} \geq \underbar{d}
  \end{align*}

  Notice that we impose per-period and cumulative limits to the amount of student loans that an
  individual can accumulate. They cannot accumulate more than $\underbar{D}$ during school and
  cannot accumulate more than $\underbar{d}$ during any given period.

  The second Bellman equation, $V^{S-}_t$, represents the college student's value of having $d_t$ in
  student loans, $k_t$ in savings, $n_t$ credits, financial state $\zeta_t$, and being of type $j$
  before making the dropout decision in period $t$

  \begin{align*}
    V_t^{S-}(d_t, k_t, n_t, \zeta_t, j) &= \begin{cases}
      E \left[ V_t^{CG}(\tilde{k}_t, d_t, \bar{S}, w_t, \varepsilon^{CG, y}_t) + U_{CG} \right]
        \quad\text{ if } n_t \geq n_{\text{grad}} \\
      E\left[ \max \{
          V_t^{CD}(\tilde{k}_t, d_t, \bar{S}, w_t, \varepsilon^{CD, y}_{t}) + U_{CD} + \pi_c \eta_D,
          V_t^S(d_t, k_t, n_t, \zeta_t, j) + \pi_c \eta_C \} \right]
        \quad\text{ else }
    \end{cases}
  \end{align*}

  The first element of this Bellman equation gives the value associated with having successfully
  accumulated $n_{\text{grad}}$ credits and becoming a graduate worker. The second element presents
  the problem associated with deciding between dropping out of college or continuing as a college
  student. The dropout/continue decision is formulated as a discrete choice where there are
  stochastic shocks to the utility of dropping out or continuing, $\pi_c \eta_D$ and $\pi_c \eta_C$,
  respectively. $\eta_D$ and $\eta_C$ are distributed as independent draws from an standard type I
  extreme value distribution with scale parameter $\pi_c$.

  There are some important notes to make at this point about the initial states of both the college
  dropout and graduate worker's problem.

  \begin{itemize}
    \item $\bar{S}$ is the number of periods that an individual who leaves college has left to repay
      their student loans and is set by the student loan repayment policy.
    \item The initial transitory income component is given by
      $\varepsilon^{s, y}_{t} \sim N(0, \sigma^2_{s, y})$
    \item The initial permanent income component, $w_t$, will be given by
      $\phi_s a + \mu n + \varepsilon^{s, w}_{t}$ --- Permanent income follows a random walk, so
      this is consistent with the process described earlier\footnote{We encourage the interested
      reader to show their equivalence.}
    \item The intial risk-free holdings of the worker are $\tilde{k}_t$, not $k_t$. We assume that
      when an individual enters the work force, they receive the present discounted value of all
      future parental transfers\footnote{
      $\tilde{k}_t = k_t + \sum_{\tau=t}^6 \left(\frac{1}{1+r} \right)^{\tau - t} z(j)$}. This, like
      the previous comment about the permanent income component, are assumptions that allows us to
      reduce the dimensionality of the value function.
  \end{itemize}

  Additionally, we assume that college students cannot stay enrolled for more than $\bar{T}_C$
  periods.


%
% Enrollment decision
%
\subsection{College Enrollment}

  After observing the idiosyncratic states, $(j, \zeta_1)$, an individual must decide whether to
  enter college or begin working as a high school worker.

  If they choose to enroll as a college student, they begin with 0 accumulated student loans, $k_1$
  savings, 0 college credits, $\zeta_1$ as their financial state, and as a type $j$ individual.

  If they choose to be a high school worker, they begin with $\tilde{k}_1 = k_1 + \sum_{t=0}^6
  \left(\frac{1}{1+r}\right)^t z(j)$ assets\footnote{As described in the college student problem,
  $z(j)$ denotes the amount of parental transfers to the student. We assume they receive the net
  present value equivalent of the parental transfers they would have received over 6 periods
  independent of college enrollment.}, $w_1 = \phi_{HS} a + \varepsilon^{HS, w}_1$ as their
  initial permanent income component, and $\varepsilon^{HS, y}_1$ as their initial transitory income
  component.

  They receive random utility shocks, $\eta_{HS}$ and $\eta_{EN}$, to the value of being a high
  school worker and the value of enrolling in college. $\eta_{HS}$ and $\eta_2$ are i.i.d. draws
  from a type I extreme value with scale parameter $\pi_E$.

  The individual's ex-ante value function is given by

  \begin{align*}
    V^{-}(\zeta_1, j) = E \left[ \max \{
      V_1^{S-}(0, k_1, 0, \zeta_1, j) + \pi_E \eta_{EN},
      V_1^{HS}(\tilde{k}_1, w_1, \varepsilon^{HS, y}_1) + U_{HS} + \pi_E \eta_{HS}
    \} \right]
  \end{align*}


% \subsection{Government}
%
%   There is a government that funds an exogenous specified amount of government expenditures, $G$,
%   and funds the student loan program.
%
%   They do this by imposing the progressive labor tax on all workers in the economy and collecting
%   the student loan repayments. As mentioned earlier, one variable of interest that arises in this
%   economy is the student loan subsidization rate. For a cohort of size $I$, let
%   $\bar{d} = \sum_{i=1}^{I} d_{i, \tau_i}$ where $\tau_i$ is the age at which individual $i$ left
%   college then the subsidization rate is given by
%
%   \begin{align*}
%     \delta_{sl} = \frac
%         {\bar{d} - \sum_{i=1}^I \sum_{t=\tau_i}^{T} \left(\frac{1}{1+r}\right)^{t - \tau_i} \rho(d_{i, t}, S_{i, t}, \tilde{y}_{i, t})}
%         {\bar{d}}
%   \end{align*}
%
%   If $\delta_{sl} > 0$ then the government gives out more money than they get in return which implies
%   that they are subsidizing educaction. If $\delta_{sl} < 0$ then the government receives more than
%   they loan out in present discounted value.
%


\subsection{Repayment plans}

  As we have previously mentioned, the two student loan plans that we'd like to discuss using our
  framework are

  \begin{enumerate}
    \item \textbf{AMR}: The standard plan which takes the loan amount at the end of college and
    amortizes the payments over 10 years.
    \item \textbf{IDR}: The REPAYE plan which takes asks individuals to pay 10\% of their disposable
    income until either the debt is repaid of they've made 20 years of payments. Any debt remaining
    after 20 years of payment is forgiven.
  \end{enumerate}

  We describe each of the repayment plans in terms of their $\rho$ and $\Gamma_S$ functions. One
  comment that applies to both sets of functions is that $S_t$ is bounded below by 0 and
  $\rho(d_t, 0, \tilde{y}_t) = 0$ $\forall d_t, \tilde{y}_t$.


  \subsubsection{Standard Repayment Plan} \label{subsubsec:amr}

    The standard repayment plan amortizes the amount of debt so that the payment is the same every
    month for 10 years. In our model, the time scale is annual, so we convert these payments into
    annual amounts. One difficult that we face is that the student loan system is incredibly complex
    in what happens during periods in which a borrower cannot make payments. This becomes slightly
    less important in our model since the time period is equivalent to a year --- very few people in
    our model cannot make their payments --- however, it still requires a decision about which
    features of the student loan program we should include. Ultimately, we choose to address this in
    what we view as the simplest solution: we assume that if an individual cannot make their
    payment, they are forced to use all of their labor income, but not personal savings, towards
    making the loan repayment. We view this as a compromise that in some ways is the ``worst
    feasible'' case for AMR\footnote{In reality, the government cannot capture 100\% of your labor
    income, but it can garnish a certain percent of your wage and make claims on your tax refunds}.
    Future work will explore what happens when we can analyze the model at a monthly frequency and
    what would happen with a more detailed model of the student loans system.

    Below we present the $\rho$ and $\Gamma_S$ functions we use in the model with the standard
    repayment plan

    \begin{align*}
      \rho(d_t, S_t, \tilde{y}_t) &=
        \begin{cases}
          \tilde{y}_t \quad\text{ if } \tilde{y}_t \leq d_t \frac{r_{sd}}{1 - \left(\frac{1}{1 + r_{sd}} \right)^{S_t}} \\
          d_t \frac{r_{sd}}{1 - \left(\frac{1}{1 + r_{sd}} \right)^{S_t}} \quad\text{ else }
        \end{cases} \\
      \Gamma_S(d_t, S_t, \tilde{y}_t) &=
        \begin{cases}
          S_t \quad\text{ if } \tilde{y}_t \leq d_t \frac{r_{sd}}{1 - \left(\frac{1}{1 + r_{sd}} \right)^{S_t}} \\
          S_t - 1 \quad\text{ else }
        \end{cases}
    \end{align*}


  \subsubsection{REPAYE Plan} \label{subsubsec:idr}

    The REPAYE plan involves less decisions because it is relatively simple. It can be described by
    the following functions:

    \begin{align*}
      \rho(d_t, S_t, \tilde{y}_t) &= \max \{ 0, \psi (\tilde{y}_t - 1.5 y_\text{poverty}) \} \\
      \Gamma_S(d_t, S_t, \tilde{y}_t) &= S_t - 1
    \end{align*}
