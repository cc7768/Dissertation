%!TEX root = ../../dissertation.tex

\section{Introduction} \label{sec:intro}

Income driven student loan repayment (IDR) plans have gained broad support in the United States,
including support by politicians from both major political parties. The key benefit of IDR is that
it provides insurance against making infeasible loan payments during times in which an individual is
experiencing unlucky labor market outcomes such as unemployment or underemployment. We focus on
potentially unintended consequences of such policies by investigating how this insurance changes
individual behavior for those considering, or already enrolled, in college. We do this by building a
detailed model of pre, intra, and post college life and we use this model to understand potential
trade-offs that arise from IDR by answering questions such as, ``Does IDR improve student
outcomes?'', ``If we see improvements, how much does it cost to achieve these improvements?'', and
``Could other policies achieve similar improvements at a lower cost?''


\subsection{Student Loans in the U.S.}

  Currently, there are two main categories of student loan repayment plans in the U.S:

  \begin{enumerate}
    \item \textbf{Time-based payments}: These plans have a predetermined sequence of payments that
    an individual is expected to make, regardless of income. These payments typically begin after
    a short grace period once a student leaves college. The plan that borrowers are automatically
    enrolled in is a time based payment plan which requires individuals to make constant monthly
    payments for 10 years. We will refer to this plan as the amortized repayment (AMR) plan. There
    are other fixed payment plans where the payment starts low and increases over time, but the
    majority of borrowers are in the AMR plan.
    \item \textbf{Income-driven Plans}: These payment plans tie the monthly payment to the amount
    that an individual earns in each month. The plans each require an individual to pay a
    pre-specified percentage of their income until their debt is paid off or until they have made a
    certain number of payments. The plans differ in the percentage of income paid each month and
    the number of payments until forgiveness. The plan we will focus on\footnote{We focus on this
    plan because it is the most recent and all new IDR debt is held in this plan.}, known as REPAYE,
    requires that an individual pay 10\% of their disposable income and for no more than 20 years.
  \end{enumerate}

  Until recently, almost all student loans were held in the AMR plan. However, the government began
  increasing support for IDR in 2008 and 2009 in response to an increase in the number of
  individuals who were struggling to make their student loan payments due to the worsening labor
  market conditions associated with the 2008 recession. In Figure~\ref{fig:idr_enrollment}, we see
  how quickly participation in IDR plans has increased over the last 5 years. In 2014, approximately
  10\% of borrowers (20\% of loan dollars) were enrolled in an IDR plan, but, by the end of 2018,
  this percentage had increased to 30\% of borrowers (50\% of loan dollars).

  As the participation rate in IDR plans has increased, so has the estimated cost of running them.
  In 2012, the Department of Education (ED) estimated that it would require a \$1.2 billion subsidy
  to offer IDR to the 2012 loan cohort, but by 2017 that estimate had been revised to over \$3
  billion dollars due to higher than expected participation and other circumstances\footnote{There
  were increases of similar magnitude in the expected subsidy for each cohort from 2009 to 2017\dots
  See Figure~\ref{fig:subsidy_original_v_revised}}. In a report\footnote{See \cite{GAO-17-22}} on
  the costs of IDR, the Government Accountability Office (GAO) raised a concern about how the
  inaccuracy of ED's forecasts impacted the ED budget. ED, while sharing this concern, highlighted
  the difficulty of forecasting these costs in their response to the GAO:

  \begin{quote}
      \dots there are a number of factors that make forecasting future IDR participation inherently
      difficult\dots entails behavioral effects that are extremely difficult to incorporate
      and project into the future
  \end{quote}

  This inherent difficulty of forecasting individual behavior motivates our approach of using a
  structural model that can account for changes in individual behavior.

\subsection{Results Preview}

  In our model, when we move from an economy that uses only AMR to an economy that only uses IDR,
  the cost of running the student loan program increases by 15\% per cohort. This increase in cost
  is accompanied by a 1 percentage point increase in the enrollment rate and almost no change in
  college completion rates.

  The reason that IDR plans have such a small effect on college enrollment is two fold: First,
  individuals who attend college have precise information about the probability with which they will
  graduate from college. Since the returns to college in this model are relatively high, those who
  have a high probability of success already choose to attend college even though they face the
  possibility of burdensome student loan repayments. Second, the probabilty of experiencing periods
  of ``burdensome student loan repayments'' is relatively low. This means that the difference in the
  value functions associated with repaying your student loan under the AMR and IDR plans is low
  enough that enrollment behavior does not significantly change. Additionally, those with a low
  probability of graduating have no reason to enroll in school because, while they would gain access
  to student loans, they would be forced to pay their yearly tuition costs without any significant
  increase in their earnings potential.

  While the effect of IDR on student outcomes is relatively small, there are non-trivial effects on
  the cost of running the student loans program. Since the probability of enrollment changes very
  little across individuals in our economy, we are examining a similar population of students ---
  The question becomes, ``Why do the same individuals take out more debt under IDR than AMR?'' The
  answer is that we have changed the incentives around debt accumulation. Under the AMR plan, all
  students expect that they will repay the debt or, if not, they will experience times in which the
  loan payments make up a large fraction of their income. This discourages students from taking on
  more debt than they expect to be able to repay. When students are allowed to repay according to an
  IDR plan, many students still expect to repay their loans, however, they know that if they were to
  experience unlucky labor market outcomes, then they wouldn't be forced to endure periods of low
  consumption and may never be expected to fully repay the loan. This creates a disconnect between
  the marginal cost and marginal benefit of an additional dollar of debt, This disconnect
  incentivizes a higher fraction of students to accumulate some debt, from 46\% under AMR to 70\%
  under IDR, and also encourages more debt accumulation among those who take out loans, from
  \$11,750 under AMR to \$13,700 under IR. Notably, these effects are largest for those who are
  least likely to graduate from college in our model.

  As with any model, we note that these numbers should be taken with a grain of salt. They depend on
  the particular structure of our model, while the model is not necessarily sensitive to
  perturbations of the parameters, it does depend on the labor income processes and the model's
  timescale in meaningful ways. We view our model as a vehicle for studying these types of higher
  education policies and these results as an initial attempt at understanding the effects that IDR
  might have on pre and intra college behavior. Future work to improve certain aspects of the labor
  market in this model would have high returns for our understanding of IDR.


\subsection{Related literature}

  There has been great interest in the effects of IDR on higher education financing. Insightful
  discussions of these programs, their proposed implementations, and some of these implications can
  be found in: \cite{ChapmanHarding1993}, \cite{Chapman1994}, \cite{BarrChapmanDeardenDynarski2018},
  and \cite{Dynarski2014}.

  We reiterate that the core argument in favor of IDR is based on the repayment burden\footnote{See
  Section~\ref{sec:example} for the precise definition of a repayment burden} imposed on borrowers
  by time based plans. According to work done in \cite{ChapmanDearden2017} and
  \cite{ChapmanLounkaew2015}, under an AMR plan, the repayment burden for individuals who earn at
  the 10th percentile in the United States with a student loan of a typical size can be in excess
  0.75. This means that, in order for certain individuals not to become delinquent on their student
  loans, they must put 75\% of their income towards paying down their student loans. IDR plans
  mitigate these type of risk by placing an upper bound on the repayment burden.

  This paper follows a literature that uses dynamic structural models to understand aspects of
  higher education including papers such as, \cite{KeaneWolpin2001}, \cite{Arcidiacono2005},
  \cite{Ionescu2009}, \cite{ChatterjeeIonescu2012}, and \cite{FerreyraGarrigaManuelli2017}.
  Recently, work done by \cite{FindeisenSachs2016}, \cite{HeijdraKindermannReijnders2017},
  \cite{Ji2017}, \cite{LuoMongey2019}, and \cite{Liu2016} have used these types of dynamic
  structural models to think about the implications of IDR and other related student loan related
  policies. We discuss some relevant work below.

  In both \cite{Ji2017} and \cite{LuoMongey2019}, the authors build a model of post-graduation labor
  markets and consider the effects that student debt has on the wages that graduates accept. In
  \cite{Ji2017}, the author finds that higher student debt leads to graduates accepting a lower wage
  quickly after graduation so that they can pay off their debt. In \cite{LuoMongey2019}, the
  authors find that, once you incorporate job satisfaction, graduates with high debt seek a higher
  wage job with a lower job satisfaction. In both papers, the authors analyze the effects associated
  with moving all loans from an AMR system to an IDR system and find that the IDR plans relieve
  some of the pressure and that the relationships they find between debt and labor market outcomes
  become smaller due to the insurance provided by IDR. Both of these papers focus only on the role
  of IDR on decisions made by college graduates and do not include endogenous enrollment or debt
  decisions which are central to the questions asked in this paper.

  \cite{HeijdraKindermannReijnders2017} uses a heterogeneous agent model with college attendance and
  human capital accumulation. In their model, rather than an IDR they use a tax levied specifically
  on those who attended college --- this differs from the standard IDR in the sense that it is
  collected indefinitely and not stopped after an individual finishes paying off their debt. They
  find that, relative to the current student loan system in the U.S., the graduate tax improves
  ex-ante welfare for all individuals in the economy. There are a few important distinctions in what
  we do. They allow individuals to choose between no higher education, an associate's degree, or a
  bachelor's degree, but do not include dropout risk or endogenous debt accumulation. Moreover, we
  believe it would be difficult to implement a pure graduate tax due to political constraints --- as
  mentioned in \cite{BarrChapmanDeardenDynarski2018}, graduate taxes effectively impose an infinite
  interest rate since they can never be repaid.

  Two other closely related papers that we think are worth mentioning are
  \cite{ChatterjeeIonescu2012} and \cite{CaiChapmanWang2018}.

  \cite{ChatterjeeIonescu2012} build a model of college attendance with the purpose of analyzing how
  the risk of dropping out and the associated negative labor market outcomes affects the decision to
  enroll in college. While these authors do not directly analyze an IDR, they do consider a related
  policy which involves forgiving the debt of college dropouts. They find that even though this plan
  induces some shirking by college students, that alleviating some of the potentially negative
  outcomes associated with dropping out encourages more individuals to attempt college and raises
  welfare by more than 2\%.

  \cite{CaiChapmanWang2018} asks very similar questions to the questions we ask in this paper in the
  context of implementing an IDR plan for student loans in China. They have panel data that includes
  the yearly earnings and education level for individuals in China. They use this data to estimate
  an income process and then explore what the implied government subsidies and repayment burdens
  would be for different repayment plans under an assumed level of student loans. The main
  differences between what they do and what we do in this paper is that (1) we consider potential
  selection effects by allowing for individuals to differ in their ability and earnings potential
  and (2) the amount of student loans accumulated in our model is endogenous whereas they focus on
  what would happen for a typical level of debt.

  Finally, the foundation of our structural model builds on \cite{HendricksLeukhina2017}. In that
  paper, the authors use a model of college enrollment and attendance to determine how predictable
  an individual's success in college is given the information that they have upon graduating from
  high school. They depend on a detailed description of the pre and intra college decisions and the
  tradeoffs induced by these decisions to answer their question. These tradeoffs are important in
  the context of our question about the cost of different student loan plans, however, in their
  model there is no labor market risk and all uncertainty is resolved once an individual leaves
  college --- however, risky labor income is crucial to the question at hand since it is one of the
  key benefits of the IDR plan and the most risky aspect of the AMR plan. To address this, we extend
  their model by explicitly modeling the working stage of life using a Bewley-Hugget-Aiyagari model
  with idiosyncratic labor market risk and incomplete financial markets.

  In fact, \cite{HendricksLeukhina2017} envision exactly this type of follow up to their work, they
  say,

  \begin{quote}
    \dots Since our model captures the distribution of risks and returns associated with entering
    college, it provides a starting point for the study of college related policies, such as income
    contingent loans or dual enrollment programs. However, the model abstracts from two features
    that may be important for policy analysis. \dots

    \dots The first feature is study effort\dots The second feature is earnings risk during the work
    phase. One motivation of making college loans income contingent is to alleviate the tight budget
    constraints of young workers who may be borrowing constrained\dots
  \end{quote}

  One additional benefit of building directly on their work is that pieces of their calibration were
  done with proprietary data that we do not have access to, and by building on their model, we can
  inherit some of these hard to determine parameter values.


\subsection{Paper Outline}

  The paper is organized as follows:

  In Section~\ref{sec:example}, we will work through a hypothetical situation involving a cohort
  of student loans to define some of the outcomes of interest. This example highlights the outcomes
  that will be of interest once we are comparing different repayment plans.

  We then describe a model similar to \cite{HendricksLeukhina2017} in Section~\ref{sec:model} and in
  Section~\ref{sec:calibration} we describe the assumptions and parameter choices for the baseline
  model.

  We discuss the outcomes of this model and the differences between AMR and IDR in
  Section~\ref{sec:results} and then conclude in Section~\ref{sec:conclusion}.
