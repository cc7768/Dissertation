%!TEX root = ../../dissertation.tex

\section{Introduction}

There is a growing interest among economists in global nonlinear solution
methods. Value function discretization is the most well-known example of a
global nonlinear solution method, but there are a variety of other iterative
and Euler-equation based methods that differ in the way they approximate,
interpolate, integrate and construct nonlinear solutions. These methods solve
economic models on grids of points that cover a relevant area of the state
space, and they generally produce more accurate solutions than perturbation
methods which typically build approximations around a single point.\footnote{
The accuracy of local perturbation solutions can decrease rapidly away from
steady state even in relatively smooth models; see \cite{KKK2011}. In models
with stronger nonlinearities, such as new Keynesian models, approximation
errors can reach hundreds of percent under empirically relevant
parameterizations; see \cite{JMM2017}. In models of labor search, a global
solution is important for accurate capturing even first moments of equilibrium
dynamics; see \cite{PZ2017}} Also, global methods are useful for solving some
models in which perturbation methods are either not applicable or their
applications are limited.\footnote{ For example, perturbation methods are not
well suitable for analyzing sovereign default problems (e.g.,
\cite{Arellano2008}) and portfolio choice problems (e.g., \cite{HK2013}). There
are perturbation-based methods that can deal with occasionally binding
constraints but they are limited to the first-order approximation. For example,
see \cite{LS2011} and \cite{GI2015}.} Finally, the recent developments of numerical
techniques for dealing with high-dimensional data has greatly extended the
class of models that can be studied with global solution methods. In
particular, it is now possible to construct global nonlinear solutions to
economic models with dozens of state variables that are intractable with
conventional global solution methods (such as value-function discretization)
due to the curse of dimensionality.\footnote{ See \cite{MM2014} for a survey of
numerical techniques that are tractable in problems with high dimensionality
and see \cite{LMM2017} for an example of constructing a global nonlinear
solution to a large-scale projection model of the Bank of Canada.}

However, global nonlinear solution methods are more difficult to automate than
perturbation methods, and their implementation requires more substantial
programming from researchers. In particular, there is still no consensus in the
profession on what software to use for implementing global solution methods,
unlike for perturbation methods where a common choice is the Dynare or IRIS platforms.
Because of this, an important aspect of the implementation of a global solution method is the
choice of programming language to use. Traditionally, Matlab is used in
economics, although some researchers have also used Fortan and C. Recently,
Python and Julia have begun to see a more widespread use in the economics
literature.\footnote{ For example, Python and Julia are used in the
\textit{Quantitative economics} lecture site of Tom Sargent and John
Stachurski. Python is used to produce the \textit{Heterogeneous Agents
Resources and toolKit} (HARK) by the team led by Christopher Carroll.}

In this paper, we pursue two goals: First, we provide a comparison between
solution methods implemented in Matlab, Python and Julia in the context of two
popular applications: a neoclassical growth model and a new Keynesian model.
Our goal is to help economic researchers choose the programming language that
is best suited to their own situation, and, if needed, help them transition
from one programming language to another. The readers can see and compare how
the same algorithm is implemented in different languages and, as a result, will
understand some of the unique aspects of each language and choose the one which
best fits their needs.\footnote{ Additionally, we provide carefully elaborated
notebooks with a description of our code and algorithms on the
\href{http://notes.quantecon.org}{QuantEcon Notebook site}\dots} The
implementation, structure and number of lines of code are similar across all
three languages.

Second, we provide a carefully documented collection of routines for Matlab,
Python and Julia that can be used for constructing global nonlinear solutions
with a special emphasis on modern techniques for problems with high
dimensionality. In particular, our code includes routines for constructing
random and quasi-random grids, low-cost monomial integration methods,
approximating functions using complete polynomials, as well as routines for
computing the numerical accuracy of solutions. Much of this code is generic in
a way which allows it to be easily portable to other applications including
problems with kinks and occasionally binding constraints. Our examples are
solved using a variety of solution techniques including: conventional policy
function iteration, conventional value function iteration, an Euler equation method,
the endogenous grid method of \cite{Carroll2006}, and several variants of the
envelope condition method of \cite{MM2015}.

\cite{AF2014} is closely related to our work. In their paper, the authors
compare 18 different programming languages by implementing the same algorithm --
value-function discretization -- to solve the stochastic neoclassical growth
model and by reporting the CPU time needed to solve the model. They find that
the choice of a programming language plays an important role in computation
time: the fastest language in their study (C++) solves the model over 450 times
faster than the slowest language (R). The present work extends \cite{AF2014} by
analyzing a variety of algorithms and also by solving a larger (in terms of the
number of state variables) model. Similar to \cite{AF2014}, we find that the
choice of a programming language may have important effects in terms of speed,
however, we do not observe such huge differences across the three languages
considered. In addition to speed comparisons, we also discuss the key features
of each language that might be useful for economists.

We build on the existing Matlab codes accompanying \cite{AMMT2016} and
\cite{MM2015}. We improve on these codes, and provide new routines that have
not been analyzed previously in the literature. First, for the neoclassical
growth model, we implement seven different methods in all three programming languages, whereas \cite{AMMT2016} is limited to Matlab implementation of the envelope condition method.
Furthermore, for the new Keynesian model, we modify the method proposed in \cite{MM2015} to operate on random and quasi-random
grids covering a fixed hypercube, instead of operating on the high probability area of the state space. The Matlab code developed in the present paper achieves an almost a 60-time speed up over the original code in \cite{MM2015},
while still producing highly accurate solutions. Our code is sufficiently fast
to be used for the estimation of a moderately-large new Keynesian model with 8
state variables. It takes us just a few seconds to construct the solution, including the model with active zero lower bound on the nominal interest rate.

Our overall experience with each language was comparable, though it is useful
to recognize that each language has it specific strengths. Both Julia and
Python are open-source which facilitates transparency and reproducibility. For
reasons to be discussed later, Julia outperforms both Matlab and Python in
algorithms that require numerical solvers or optimizers. Python benefits from
an active community and strong package ecosystem which makes finding the right
tools easy. Finally, Matlab benefits from extensive documentation, technical
support, and various built in toolboxes. Ultimately, the best choice of
programming language depends on various features including: model size,
solution complexity, ease of writing, and co-author preferences. Regardless of
the programming language chosen, our experience shows that transitioning
between each of the three languages we consider should not require a
substantial learning effort.

The rest of the paper is organized as follows. In Section 2, we describe the
three languages, Matlab, Python and Julia. In Section 3, we outline seven
algorithms we implement to solve a variant of the standard neoclassical
stochastic growth model. In Section 4, we present an algorithm for solving a
medium-scale new Keynesian model with a zero lower bound on nominal interest
rates and describe the results of our analysis. Finally, in Section 5 we
conclude.
