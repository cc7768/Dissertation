%!TEX root = ../../dissertation.tex

\section{Conclusion}

We have described and implemented seven algorithms to solve the stochastic
neoclassical growth model, and we have described and analyzed a modification of
an algorithm that has been previously used to globally solve new Keynesian
models. The implementation has provided us a \textquotedblleft
laboratory\textquotedblright \ to explore some of the strengths and weaknesses
of the three languages we consider. One insight we gained was that for some
algorithms, the choice of programming language has little effect on
performance, but for others, there are important speedups that can be obtained.
In particular, if an algorithm depends heavily on numerical optimization or
root-finding then it is likely that Julia will provide significant speedups,
but may come at the cost of ensuring that your code provides enough information
to the compiler.

We hope we have brought some clarity in the trade-off between these three
programming languages which will first, provide information to economists who
are thinking about which language is the best fit for them; and second, provide
a clear description, examples, and tools which can facilitate other researchers
implement similar algorithms for their own models.
