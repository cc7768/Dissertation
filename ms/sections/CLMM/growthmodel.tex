%!TEX root = ../../dissertation.tex

\section{Comparison using a neoclassical growth model}

The neoclassical growth model is often used as a benchmark to measure the
effectiveness of a solution method. We keep with this tradition and analyze the
stochastic neoclassical growth model with seven alternative solution methods,
which are outlined in \cite{AMMT2016}. Each method is implemented in Julia,
Matlab, and Python which allows us to see how the performance of different
solution methods may vary by programming language.

\subsection{The model}

We consider a dynamic programming problem of finding the value function,
$V$, that solves the Bellman equation,

\begin{align}
  V \left(k, z\right) &= \max_{c, k^{\prime }} u\left(c\right) +\beta E\left[ V\left( k^{\prime },z^{\prime }\right) \right] \label{vf} \\
  \text{s.t.\quad}k^{\prime } &= \left( 1-\delta \right) k + z f\left( k\right) - c \label{bc} \\
  \ln z^{\prime} &= \rho \ln z + \varepsilon^{\prime },\qquad \varepsilon^{\prime } \sim \mathcal{N}\left( 0,\sigma ^{2}\right), \label{ts}
\end{align}

where $k$, $c$ and $z$ are capital, consumption and productivity level,
respectively; $\beta \in \left( 0,1\right) $; $\delta \in \left( 0,1\right] $ ;
$\rho \in \left( -1,1\right) $; $\sigma \geq 0$; the utility and production
functions, $u$ and $f$, respectively, are strictly increasing, continuously
differentiable and strictly concave. The primes on variables denote next-period
values, and $E\left[ V\left( k^{\prime },z^{\prime }\right) \right] $ is an
expectation conditional on state $\left( k,z\right) $.

\paragraph{Optimality conditions.}

The first order condition (FOC) and envelope condition (EC) of the problem
(\ref{vf})-(\ref{ts}), respectively, are

\begin{equation}
  u^{\prime}\left(c\right) = \beta E \left[ V_{1}\left(k^{\prime }, z^{\prime}\right) \right] \label{f2}
\end{equation}

\begin{equation}
  V_{1}\left(k, z\right) = u^{\prime }\left( c\right) \left[ 1-\delta + zf^{\prime }\left( k\right) \right]. \label{f3}
\end{equation}%

By combining (\ref{f2}) and (\ref{f3}), we obtain the Euler equation%

\begin{equation}
  u^{\prime }\left(c\right) = \beta E\left[ u^{\prime }\left( c^{\prime}\right) \left[ 1-\delta +z^{\prime }f^{\prime }\left( k^{\prime} \right) \right] \right]. \label{f23}
\end{equation}

\paragraph{Parameterization and implementation details.}

We parameterize the model (\ref{vf})-(\ref{ts}) by using a Constant Relative
Risk Aversion (CRRA) utility function, $u\left( c\right) =\frac{c^{1-\gamma
}-1}{1-\gamma }$, and a Cobb-Douglas production function, $f\left( k\right)
=Ak^{\alpha }$. We choose parameters to be set at $A=\frac{1/\beta -(1-\delta
)}{\alpha }$, $\alpha =1/3$, $\beta =0.99$, $\delta =0.025$, $ \rho =0.95$ and
$\sigma =0.01$. As a solution domain, we use a rectangular, uniformly spaced
grid of $10\times 10$ points for capital and productivity between 0.9 and 1.1
for both variables.

We integrate by using a $10$-node Gauss-Hermite quadrature rule and approximate
the policy and value functions by using complete ordinary polynomials up to
degree 5. As an initial guess, we use a linear approximation to the capital
policy function. To solve for polynomial coefficients, we use fixed-point
iteration.

All computations are performed using Julia v0.6.0, Matlab version 9.2.0
(R2017a), and Python 3.6 on a MacBook Pro with a 2.7 GHz Intel Core i7
processor and 16 GB of RAM. For Julia and Python, the particular package
versions can be found in the corresponding notebooks submitted on QuantEcon's Notebook site\footnote{The
\href{http://notes.quantecon.org/submission/5b5f71779cd7f00015be6350}{Matlab notebook},
\href{http://notes.quantecon.org/submission/5b5f70db9cd7f00015be634e}{Python notebook},
\href{http://notes.quantecon.org/submission/5b5f711d9cd7f00015be634f}{Julia notebook}}.

\subsection{Value iterative methods}

We analyze three value iterative algorithms for solving the model: (1)
conventional value function iteration method analyzed in e.g., \cite{AF2014} ;
(2) a variant of envelope condition method (ECM) of \cite{MM2013} that finds a
solution to Bellman equation by iterating on value function; and (3) endogenous
grid method (EGM) of \cite{Carroll2006}. We present a brief description of each
algorithm below. A detailed description of each algorithm is included in
Appendix A.

\subsubsection{Conventional Value Function Iteration}

Conventional VFI constructs the policy rule for consumption by combining FOC
(\ref{f2}) and budget constraint (\ref{bc}):

\qquad \newline
{\small
\begin{tabular}{l}
\hline \hline
\textbf{Algorithm 1. Conventional VFI} \\ \hline
Given $V$, for each point $\left( k,z\right) $, define the following recursion: \\
\quad i). Numerically solve for $c$ satisfying $u^{\prime }\left(c\right) =\beta E \left[ V_{1}\left( \left( 1-\delta \right) k+zf\left( k\right) -c,z^{\prime }\right) \right] $. \\
\quad ii). Find $k^{\prime }=\left( 1-\delta \right) k+zf\left( k\right) -c$. \\
\quad iii). Find $\widehat{V}\left( k,z\right) =u\left( c\right) +\beta E \left[ V\left( k^{\prime },z^{\prime }\right) \right] $. \\
Iterate on i)-iii) until convergence $\widehat{V}=V$. \\ \hline \hline
\end{tabular}
}

\qquad \newline

\subsubsection{Envelope Condition Method}

ECM, proposed in \cite{MM2013}, finds consumption from envelope
condition (\ref{f3}) and budget constraint (\ref{bc}). In this specific model,
the consumption function can be constructed analytically:

\qquad \newline

{\small
\begin{tabular}{l}
\hline \hline
\textbf{Algorithm 2. ECM-VF} \\ \hline
Given $V$, for each point $\left( k,z\right) $, define the following recursion: \\
\quad i). Find $c=u^{\prime -1}\left[ \frac{V_{1}\left( k,z\right) }{1-\delta +zf^{\prime }\left( k\right) }\right]$. \\
\quad ii). Find $k^{\prime }=\left( 1-\delta \right) k+zf\left( k\right) -c$. \\
\quad iii). Find $\widehat{V}\left( k,z\right) =u\left( c\right) +\beta E \left[ V\left( k^{\prime },z^{\prime }\right) \right]$. \\
Iterate on i)-iii) until convergence $\widehat{V}=V$. \\ \hline \hline
\end{tabular}
}

\qquad \newline

\subsubsection{Endogenous Grid Method}

Finally, EGM of \cite{Carroll2006} constructs a grid on $\left( k^{\prime
},z\right) $ by fixing the future endogenous state variable $k^{\prime }$ and
by treating the current endogenous state variable $k$ as unknown; see also
\cite{BF2007} for a discussion of the EGM method. Since $k^{\prime }$ is fixed,
EGM computes $E\left[ V_{1}\left( k^{\prime },z^{\prime }\right) \right] $
up-front and thus can avoid costly interpolation and approximation of
expectation in a root finding procedure.

\qquad \newline

{\small
\begin{tabular}{l}
\hline \hline
\textbf{Algorithm 3. EGM of Carroll (2005)} \\ \hline
Given $V$, for each point $\left( k^{\prime },z\right) $, define the
following recursion: \\
\quad i). Find $c=u^{\prime -1}\left \{ \beta E\left[ V_{1}\left( k^{\prime}, z^{\prime }\right) \right] \right \}$. \\
\quad ii). Solve for $k$ satisfying $k^{\prime }=\left( 1-\delta \right)k + zf\left(k\right) - c$. \\
\quad iii). Find $\widehat{V}\left( k,z\right) =u\left( c\right) +\beta E \left[ V\left( k^{\prime },z^{\prime }\right) \right]$. \\
Iterate on i)-iii) until convergence $\widehat{V}=V$. \\ \hline \hline
\end{tabular}%
}

\qquad \newline

In Step ii) of EGM, we still need to find $k$ numerically. However, for the
studied model, \cite{Carroll2006} shows a change of variables that makes it
possible to avoid finding $k$ numerically on each iteration (except of the
very last iteration). In order to streamline our code, we do not use this
change of variables so the running time of our version of the EGM method
will be considerably larger than for the version of the method in \cite
{Carroll2006}. However, this fact does not distort our comparison analysis
since we use the same implementation in all three languages.

\subsubsection{Comparison results of Matlab, Python and Julia for value iterative methods}

The results from our comparison of the three iterative methods are in Table
\ref{clmm:Table1}. Conventional VFI is the most time consuming in all three
languages because it requires us to find the root of (\ref{f2}) for each $(k,
z)$. Given $(k, z)$, each evaluation of equation (\ref{f2}) requires the computer
to compute conditional expectation by interpolating over the $t+1$ values.
Numerical root finders must do this repeatedly. The reason EGM performs better
than VFI is that it is easier to solve equation (\ref{f2}) with respect to $c$
given $\left(k^{\prime}, z\right)$ than to solve it with respect to $c$ given
$(k, z)$ because we only need to evaluate conditional expectation once if
we know $k^{\prime}$. The ECM method requires no numerical solver which is why
it is so efficient relative to the other two methods.\footnote{A version of the
envelope condition argument is used in \cite{AHLLM2017} to construct a new
class of fast and efficient numerical methods for solving dynamic economic
models in continuous time.}

Julia produces solutions between 5 and 50 times faster than Matlab and Python
for VFI and EGM. The reasons Julia performs so much better on these two
algorithms is because the entire language is built on JIT compilation. This
means that the repeated calls to the Julia objective
function are cheap (relative to an interpreted language like Matlab or Python)
because these subsequent function calls execute already compiled machine code.

Our results for the ECM show similar performance for all three languages. The
main reason for this is that the method does not require a non-linear solver to
compute the policy for consumption and computations can be vectorized. The
vectorization puts the languages on more similar footing because they all end
up calling out to the same high performant BLAS routines implemented in C or
Fortran.


\subsection{Policy iterating methods}

We analyze four algorithms that construct policy functions for the standard
neoclassical stochastic growth model: Algorithm 4 is a conventional policy iteration (PI)
method, e.g., \cite{SR2004}; Algorithm 5 is a variant of ECM that implements policy
iteration; Algorithm 6 is a version of ECM that solve for derivative of value function
instead of value function itself; Algorithm 7 an Euler equation algorithm. Again we
give a brief description of each algorithm but a more detailed description of
these algorithms is provided in Appendix A.

\subsubsection{Conventional Policy Iteration}

Conventional policy iteration constructs a solution to the Bellman equation by
iterating on the consumption function using FOC (\ref{f2}).

\qquad \newline

{\small
\begin{tabular}{l}
\hline \hline
\textbf{Algorithm 4. Conventional PI} \\ \hline
Given $C$, for each point $\left( k,z\right) $, define the following
recursion: \\
\quad i). Find $K\left(k, z\right) = \left( 1 - \delta \right)k + z f\left(k\right) - C\left( k,z\right)$ \\
\quad ii). Solve for $V$ satisfying $V\left(k, z\right) = u\left(C\left(k, z\right) \right) + \beta E\left[V\left(K\left(k, z\right), z^{\prime} \right) \right]$ \\
\quad iii). Find $\widehat{C}$ satisfying $u^{\prime }\left( \widehat{C}\left( k,z\right) \right) =\beta E\left[ V_{1}\left( \left( 1-\delta \right)k + z f\left(k\right) - \widehat{C}\left(k, z\right), z^{\prime}\right) \right]
$ \\
Iterate on i)-iii) until convergence $\widehat{C}=C$. \\ \hline \hline
\end{tabular}%
}

\qquad

\subsubsection{ECM Policy Iteration}

ECM-PI the variant of ECM that performs PI instead of VFI. It constructs a
solution to the Bellman equation by iterating on the consumption function using
EC (\ref{f3}).

\qquad \newline

{\small
\begin{tabular}{l}
\hline \hline
\textbf{Algorithm 5. ECM-PI} \\ \hline
Given $C$, for each point $\left( k,z\right) $, define the following
recursion: \\
\quad i). Find $K\left(k, z\right) = \left( 1 - \delta \right) k+zf\left(k\right) - C\left(k, z\right)$ \\
\quad ii). Solve for $V$ satisfying $V\left(k, z\right) = u\left( C\left(k, z\right) \right) + \beta E\left[ V\left(K\left(k, z\right), z^{\prime}\right) \right]$ \\
\quad iii). Find $\widehat{C}\left(k, z\right) = u^{\prime -1}\left[\frac{V_{1}\left(k, z\right)}{1 - \delta + z f^{\prime }\left(k\right)}\right]$ \\
Iterate on i)-iii) until convergence $\widehat{C}=C$. \\ \hline \hline
\end{tabular}%
}

\qquad

\subsubsection{Derivative Policy Iteration}

The ECM analysis also suggests a useful recursion for the derivative of value
function. We first construct consumption function $C\left(k, z\right)$
satisfying FOC (\ref{f2}) under the current value function

\begin{align}
\beta E\left[V_{1}\left( k^{\prime },z^{\prime }\right) \right] = u^{\prime }\left(C\left(k, z\right) \right), \label{vk4}
\end{align}

and we then use (\ref{f3}) to obtain the derivative of value function for next
iteration. This leads to a solution method that we call ECM-DVF.

The main difference between ECM-DVF from the previously studied ECM-VF consists
in that we iterate on $V_{1}$ without computing $V$ on each iteration. We only
compute $V$ at the very end, when both $V_{1}$ and the optimal policy functions
are constructed. Again, neither numerical maximization nor a numerical solver
is necessary under ECM-DVF but only direct calculations. Similarly to PI
methods, Euler equation methods do not solve for value function but only for
decision (policy) functions. One possible decision function is a derivative of
value function. Thus, the ECM-DVF recursion (\ref{vk4}) can be also viewed as
an Euler equation written in terms of the derivative of value function.

\qquad \newline

{\small
\begin{tabular}{l}
\hline \hline
\textbf{Algorithm 6. ECM-DVF} \\ \hline
Given $V_{1}$ for each point $\left( k,z\right) $, define the following recursion: \\
\quad i). Find $c = u^{\prime -1}\left[ \frac{V_{1}\left(k, z\right)}{1 - \delta +z f^{ prime}\left(k\right) }\right]$ \\
\quad ii). Find $k^{\prime }=\left( 1-\delta \right) k+zf\left( k\right) -c$ \\
\quad iii). Find $\widehat{V}_{1}\left(k, z\right) =\beta \left[ 1-\delta +z f^{\prime }\left(k\right) \right] E \left[V_{1}\left(k^{\prime}, z^{\prime }\right) \right]$ \\
Iterate on i)-iii) until convergence $\widehat{V}_{1}=V_{1}$ \\
Given the converged policy functions, find $V$ satisfying $V\left(k, z\right) = u\left(c\right) + \beta E \left[V\left(k^{\prime }, z^{\prime}\right) \right]$ \\ \hline \hline
\end{tabular}
}

\subsubsection{Euler Equation Methods}

Policy iteration has similarity to Euler equation methods; see \cite
{Judd1998}, \cite{Santos1999} for a general discussion of such methods. Euler
equation methods approximate policy functions for consumption $ c=C\left(
k,z\right) $, capital $k^{\prime }=K\left( k,z\right) $ (or other policy
functions) to satisfy (\ref{bc}), (\ref{ts}) and (\ref{f23}). Below, we provide
an example of Euler equation method (many other recursions for such methods are
possible).

\qquad \newline

{\small
\begin{tabular}{l}
\hline \hline
\textbf{Algorithm 7. Euler equation algorithms} \\ \hline
Given $C\left( k,z\right) $, for each point $\left( k,z\right) $, define the
following recursion: \\
\quad i). Find $K\left(k, z\right) =\left(1 - \delta\right) k + z f\left(k\right) -C\left( k,z\right)$ \\
\quad ii). Find $c^{\prime} = C\left(K\left(k, z\right), z^{\prime}\right)$ \\
\quad iii). Find $\widehat{C}\left(k, z\right) = u^{\prime -1}\left \{\beta E \left[ u^{\prime}\left(c^{\prime}\right) \left(1 - \delta + z^{\prime} f^{\prime}\left(k^{\prime}\right) \right) \right] \right \}$ \\
Iterate on i)-iii) until convergence $\widehat{C}=C$. \\ \hline \hline
\end{tabular}
}

\subsubsection{Comparison results of Matlab, Python and Julia for policy iterative methods}

In Table \ref{clmm:Table2}, we see that conventional PI performs the worst because it relies on a
numerical solver, but, as with the value iterative methods, this reliance is less problematic for
Julia than for Matlab or Python.

Overall, we found that the speed of most algorithms was comparable across all three languages. The
exception was that Julia performed significantly better than the other programs whenever there was
any numerical optimization or root-finding involved. However, these speedups came at the cost of
spending a (relatively small) amount of extra time being careful about how the code was written to
ensure that the Julia compiler is able to infer the type of each variable in the most time-intensive
parts of the program.
