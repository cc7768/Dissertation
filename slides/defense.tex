\documentclass[10pt]{beamer}

  % Math Packages
  \usepackage{amsmath}
  \usepackage{amsthm}
  \usepackage{booktabs}
  \usepackage{mathtools}

  % Graphics Packages
  \usepackage{graphicx}
  \usepackage[export]{adjustbox}

  % Style and util packages
  \usepackage{caption}
  \usepackage{dirtytalk}

  % Bibliography
  \usepackage{hyperref}
  \usepackage{natbib}
  \bibliographystyle{ecta}

  % Theme Settings
  \usetheme{metropolis}
  \usefonttheme{professionalfonts}

  \hypersetup{
    % colorlinks=true,  % false: boxed links; true: colored links
    % linkcolor=red,  % color of internal links (change box color with linkbordercolor)
    citecolor=blue,  % color of links to bibliography
    % urlcolor=red  % color of external links
  }

  % Colors
  \definecolor{myred}{rgb}{1.0, 0.11, 0.0}
  \definecolor{mygreen}{rgb}{0.13, 0.55, 0.13}

\title{Dissertation Defense}
\author{Chase Coleman}
\institute{NYU Stern}
\date[]{\today}

\begin{document}

% Title Slide
\begin{frame}
  \thispagestyle{empty}
  \titlepage
\end{frame}

% --------------------------------------- %
% Introduction
% --------------------------------------- %
\begin{frame} \frametitle{What have I learned?}

  Three Chapters:

  \vspace{0.5cm}

  \begin{enumerate}
    \item Pareto weights as wedges in two-country models (with Dave Backus, Axelle Ferriere,
          and Spencer Lyon)
    \item Global solution methods for macroeconomic models (with Spencer Lyon, Lilia Maliar, and
          Serguei Maliar)
    \item Cost of income-driven repayment for student loans
  \end{enumerate}

\end{frame}

% -------------------------------------------- %
% Pareto weights as wedges in two-country models
% -------------------------------------------- %
\section{Pareto weights as wedges in two-country models}

  \begin{frame} \frametitle{Introduction}

    \textbf{Environment}: Typical two country, two (tradeable) goods, infinite horizon model (a la
    Backus Kehoe Kydland 1994). Incorporate recursive preferences following Colacito Croce
    2011/2013, Tretvoll 2011/2015

    \vspace{0.5cm}

    \textbf{Want}: Understand the stochastic process that governs the relative pareto weights in a
    two country economy with recursive preferences

  \end{frame}

  \begin{frame} \frametitle{Pareto weight process}

    \textbf{Stochastic process for pareto weights}:

    \begin{align*}
      \log \lambda_{t+1}^{*} &= \log \lambda_{t}^* + (\rho - \alpha) \left[ log\left( \frac{U_{t+1}}{\mu(U_{t+1})} \right) - \log \left( \frac{J_{t+1}}{\mu(J_{t+1})} \right) \right] \\
    \end{align*}

    \vspace{-0.15cm}

    \textbf{Additive Case}: $\alpha = \rho$ implies $\log \lambda_{t+1}^* = \log \lambda_t^*$

    \vspace{0.25cm}

    \textbf{Rewrite}: Stochastic process can be rewritten as white noise plus a risk adjustment ---
    If no risk adjustment then process would have unit root.

  \end{frame}

  \begin{frame} \frametitle{Stable pareto weights}

    In one good models pareto weights are unstable and one agent eventually consumes everything. It
    turns out that in two good model that is not true (but they are extremely persistent)!

    \begin{center}
      \begin{figure}[htb]
        \includegraphics[width=0.8\textwidth]{../ms/images/BCFL/policy_at_ss_one_axis.pdf}
      \end{figure}
    \end{center}

  \end{frame}

  \begin{frame} \frametitle{Backus-Smith puzzle}

    Backus-Smith puzzle: Theory (with additive preferences) predicts negative relationship between
    exchange rate and relative consumption but approximately 0 in data

    \begin{figure}[htb]
      \includegraphics[width=0.8\textwidth]{../ms/images/BCFL/con_v_fxr_alpha9.pdf}
    \end{figure}

  \end{frame}

% --------------------------------------------------- %
% Global solution algorithms for macroeconomic models
% --------------------------------------------------- %
\section{Global Solution methods for macroeconomic models}

  \begin{frame} \frametitle{Global solutions}

    Global solutions allow more accurate characterizations of model dynamics than perturbation
    methods --- Many questions require this accuracy:

    \vspace{0.25cm}

    \begin{itemize}
      \item Labor search models: Petrosky-Nadeau Zhang (Forthcoming)
      \item Sovereign default models: Arellano et al (2016)
      \item Certain classes of NK models: Judd et al 2017
    \end{itemize}

  \end{frame}

  \begin{frame} \frametitle{Epsilon-distinguishable sets}

    Maliar Maliar 2015 introduces \textit{Epsilon-distinguishable sets} for creating non-tensor
    grids to solve high dimensional state problems

    \begin{figure}
      \includegraphics[width=0.8\textwidth]{../ms/images/CLMM/Figure1-eps-converted-to.pdf}
    \end{figure}

  \end{frame}

  \begin{frame} \frametitle{Pseudo-random grids}

    \textit{Epsilon-distinguishable sets} requires simulation of model\dots If we have a rough idea
    of where states live, can skip simulation step by drawing quasi-random grids in relevant
    hypercube. Removing simulation step leads to between 1 and 2 orders of magnitude faster
    solution.

    \begin{figure}
      \includegraphics[width=0.8\textwidth]{../ms/images/CLMM/Figure2-eps-converted-to.pdf}
    \end{figure}

  \end{frame}

% --------------------------------------------------- %
% Cost of income-driven student loans
% --------------------------------------------------- %
\section{Cost of income-driven student loans}


  \begin{frame} \frametitle{Introduction}

    \say{
      Students should not be asked to pay more on their loans than they can afford\dots That's why
      under my student loan program, we would cap repayment for an affordable portion of a borrower's
      income, 12.5 percent\dots If borrowers work hard and make their full payments for 15 years,
      we'll let them get on with their lives --- DJT
    }

  \end{frame}

  \begin{frame} \frametitle{U.S. student loan repayment plans}

    \begin{table}
      {\small
      \hspace*{-0.5cm}\begin{tabular}{lll}
        \toprule
        \textbf{Name} & \textbf{Payment} & \textbf{Forgiveness} \\
        \midrule
        {\color{myred} AMR} & Amortized over 10 years & None \\
        {\color{myred} Graduated} & Fixed schedule which increases over 10 years & None \\
        {\color{mygreen} IBR} & 15\% of disposable income (up to AMR) & 25 years \\
        {\color{mygreen} REPAYE} & 10\% of disposable income & 20 years \\
        {\color{mygreen} Trump Proposal} & 12.5\% of disposable income & 15 years \\
        \bottomrule
      \end{tabular}
      }
    \end{table}

  \end{frame}

  \begin{frame} \frametitle{IDR expanding}

     \begin{figure}
       \includegraphics[width=\textwidth]{../ms/images/StudentLoans/idr_enrollment_ts.pdf}
       \caption*{\tiny{U.S. Department of Education, Federal Student Aid Data Center, Federal Student Loan Portfolio.}}
     \end{figure}

  \end{frame}

  \begin{frame} \frametitle{Hard to determine cost}

     \begin{figure}
       \includegraphics[width=\textwidth]{../ms/images/StudentLoans/GovSubsidy_Original_v_Revised.pdf}
       \caption*{\tiny{GAO analysis of the U.S. Department of Educations' 2011-2017 budget estimates; \cite{GAO-17-22}}}
     \end{figure}

  \end{frame}

  \begin{frame} \frametitle{Hard to determine cost}

    \say{
        \dots there are a number of factors that make forecasting future IDR participation
        inherently difficult\dots it entails behavioral effects that are extremely difficult to
        incorporate and project into the future --- Department of Education response to
        Goverment Accountability Office
    }

  \end{frame}

  \begin{frame} \frametitle{Paper counterfactual}

    In this paper, I run the following counterfactual:

    \vspace{0.25cm}

    Consider two economies

    \begin{enumerate}
      \item All student loans held in AMR repayment plan
      \item All student loans held in IDR repayment plan
    \end{enumerate}

    Will determine how much it costs for the government to support IDR relative to AMR

  \end{frame}

  \begin{frame} \frametitle{Government subsidy}

      Let $\Delta_{SL}$ be the subsidy required to keep student loans program solvent:

      \vspace{0.5cm}

      $$\Delta_{SL} = \sum_i (d_i - X_i)$$

      \vspace{0.5cm}

      where $d_i$ is debt accumulated by individual $i$ and $X_i$ is the present discounted value
      of the payments they make

  \end{frame}

  \begin{frame} \frametitle{Government subsidy decomposition}

      Decompose $\Delta_{SL}$ by

      \vspace{0.5cm}

      \begin{align*} \label{eq:gov_subsidy}
        \Delta_{SL} &=
            \underbrace{\frac{\sum_i (d_i - X_i)}{\sum_i d_i}}_{\text{subsidy rate}} \times
            \underbrace{\frac{\sum_i d_i}{N_d}}_{\text{average debt}} \times
            \underbrace{\frac{N_d}{N_e}}_{\text{fraction in debt}} \times
            \underbrace{\frac{N_e}{N}}_{\text{enrollment rate}} \times
            N
      \end{align*}

  \end{frame}

  \begin{frame} \frametitle{What we will learn today}

    We find that going from an economy with only AMR to an economy with only IDR results in

    \vspace{0.5cm}

    \begin{itemize}
      \item A 1 pp increase in enrollment rate
      \item 24 pp increase in percent of students with debt
      \item 17\% increase in average student loan size
      \item 50\% increase in subsidy rate
    \end{itemize}

    \vspace{0.5cm}

    These result in a 15\% increase in the cost of running the student loans program

  \end{frame}

  \begin{frame} \frametitle{Model outline: idiosyncratic type}

    All students begin as high school graduates. Draw three idiosyncratic states:

    \begin{itemize}
      \item Ability level ($a$): Unobservable to individuals and will affect the probability with
      which they pass classes in college and their labor earnings.
      \item Type ($j$): Type consists of 4 components $(m, k, q, z)$ --- $m$ is a signal about an
      individual's ability level, $k$ is the initial risk-free holdings, $q$ is the cost of college,
      and $z$ is they yearly parental transfer.
      \item Financial state ($\zeta_1$): Stochastic component of the cost of college and college
      work opportunities.
    \end{itemize}

    Additionally, assume econometrician cannot observe $m$, but rather observes

    $$\text{GPA} = m + \varepsilon_{\text{GPA}}$$

  \end{frame}

  \begin{frame} \frametitle{Model outline}

    \begin{figure}
      \begin{center}
       \includegraphics[width=\textwidth]{../ms/images/StudentLoans/LifeCycleChart.png}
      \end{center}
    \end{figure}

  \end{frame}

  \begin{frame} \frametitle{Enrollment effects}

    The probability of enrollment is given by:

    $$\text{prob}(\text{enroll}) = \frac{\exp(V^{S})}{\exp(V^{S}) + \exp(V^{HS})}$$

    $V^{HS}$ unaffected by changes in repayment plan which means all changes must come through
    changes in $V^{S}$. $V^{S}$ is itself only indirectly affected by changes in $V^{CD}$ and
    $V^{CG}$

  \end{frame}

  \begin{frame} \frametitle{Enrollment effects}

    \begin{figure}
      \begin{center}
       \includegraphics[width=\textwidth]{../ms/images/StudentLoans/vf_vs_income_probs.pdf}
      \end{center}
    \end{figure}

  \end{frame}

  \begin{frame} \frametitle{Debt effects}

    \begin{figure}
      \begin{center}
       \includegraphics[width=\textwidth]{../ms/images/StudentLoans/DebtByGPA.pdf}
      \end{center}
    \end{figure}

  \end{frame}

  \begin{frame} \frametitle{Debt effects}

    \textbf{Thought experiment}:

    Imagine that an individual gets an information shock at beginning of the last period of their
    college degree.

    \vspace{0.1cm}

    They find out that they will experience low income for 20 years and, at their current student
    loan level, will experience loan forgiveness. What are implications for student loan
    accumulation this period?

    \vspace{0.1cm}

    If forgiveness, $\frac{\partial V^{CD}}{\partial d_t} = 0$ implies higher marginal utility for
    debt today --- Take out maximum debt level possible.

    \vspace{0.3cm}

    Lowest quartile increases debt the most because highest probability of forgiveness

  \end{frame}

  \begin{frame} \frametitle{Expected repayment by loan size}

    \begin{figure}
      \begin{center}
       \includegraphics[width=\textwidth]{../ms/images/StudentLoans/expected_repayment.pdf}
      \end{center}
    \end{figure}

  \end{frame}

  \begin{frame} \frametitle{Government subsidy rate}

    \begin{table}[H]
      \begin{tabular}{lll}
        \toprule
        & \textbf{AMR} & \textbf{IDR} \\
        \midrule
        Quartile 1 & -0.13 & 0.11 \\
        Quartile 2 & -0.13 & -0.01 \\
        Quartile 3 & -0.13 & -0.07 \\
        Quartile 4 & -0.13 & -0.11 \\
        \bottomrule
      \end{tabular}
    \end{table}

  \end{frame}

  \begin{frame} \frametitle{Conclusion and next steps}

    IDR is expensive --- More targeted policies to reduce student loan repayment risk may be more
    successful.

    For example, one policy that reduces overall government subsidy in IDR to the level from AMR by
    offering IDR only to those in top 3 GPA quartiles.

    \vspace{0.5cm}

    Occupation choice? Major choice? Higher frequency model to capture more labor market risk?

  \end{frame}

\end{document}
